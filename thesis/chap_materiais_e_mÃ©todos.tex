\chapter{Materiais e Métodos}

\section{Proposta de Ferramenta}

Dois equipamentos serão utilizados para construção da ferramenta de coleta sincronizada: Mindwave Mobile 2 da Neurosky Inc. e o
GP3 da Gazepoint Inc., para a coleta de EEG e ET, respectivamente. Um código será construído de forma a permitir a coleta 
síncrona do movimento ocular e das ondas cerebrais pelos dois equipamentos comerciais. 
No presente capítulo as ferramentas serão apresentadas no que diz respeito às suas principais características, 
e especificidades do código de coleta serão apresentadas.

\subsection{Mindwave Mobile II}
O equipamento (representado na figura 6.1) possui um eletrodo de coleta e um eletrodo de referência que ficam posicionados 
acima da sobrancelha esquerda e na orelha esquerda do participante, respectivamente. 
A posição do eletrodo de coleta em relação ao sistema de referência de posição de eletrodos (10-20), é o FP1, 
correspondendo a região Frontopolar 1. A coleta de dados do aparelho se dá por conexão via bluetooth e 
funciona em computadores Mac, Windows ou celulares Androids ou iOS, disponíveis em um raio de 10 metros. 
Ele coleta ondas cerebrais variando entre 3 e 100Hz, com uma frequência de 512Hz (NeuroSky Inc., 2015). 
O aparelho automaticamente distingue os dados coletados em ondas alfa, beta, gama, teta e delta; 
além de coletar informações subjetivas no formato de medidas de atenção e meditação, 
por meio de um algoritmo de reforço de aprendizado não disponibilizado ao publico. 
Também mede a ativação muscular próxima ao eletrodo para estimar a qualidade do sinal. 
O MindWave Mobile filtra interferência elétrica e converte o sinal detectado pelo eletrodo em sinal digital. 
O chip que faz o filtro e conversão se chama ThinkGear, e permite a filtragem de ruído para interferência ativação muscular (EMG) e 50/60Hz de corrente alternada. 

\subsection{Gazepoint GP3}
O GP3 é um equipamento comercial de coleta do movimento dos olhos, 
fabricado pela Gazepoint Inc (representado na figura 6.2). Possui software próprio para análise dos dados, 
além de ser possível realizar coleta de dados com linguagens de programação open-source. 
Coleta em uma frequência de 60Hz, com 0.5 a 1 grau de acurácia do ângulo visual e cinco a nove 
pontos para ajuste do ponto focal (Gazepoint Inc.). Consiste em dois cabos USBs, um tripé próprio e a câmera de gravação do olhar (Figura 1). 

O GP3 funciona emitindo uma luz infravermelha (IR) 
diretamente nos olhos do participante e captando a reflexão da luz para localizar o ponto focal ao longo do tempo. 
Permite coletar a direção do olhar, número de fixações, tempo até a primeira fixação, taxa de piscadas,
 duração de piscadas, diâmetro da pupila, tempo de duração do olhar em um determinado ponto focal, 
 objetos observados em uma imagem, entre outros (Gazepoint Inc.).

\subsection{}