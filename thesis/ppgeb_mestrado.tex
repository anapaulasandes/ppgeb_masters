\documentclass[a4paper, 12pt]{ppgeb}

% |--- Títulos, autor, banca ---|----------------------{{{
% Autor:
% Substituta  as informações nos comandos a seguir, até a linha começando
% com \membrobancaexterno.
% Em \title: título na forma principal, como aparecerá em algumas páginas
% Em \tituloficha: título como aparecerá na ficha catalográfica; idêntico
% ao anterior, mas com possíveis quebras manuais de linha (usar \\ quando
% necessário, para ajustar as mudanças de linha na ficha catalográfica).
% Em   \titulocapaA,   \titulocapaB,   \titulocapaC:  título para a capa,
% dividido em no  máximo 3  linhas (coloque uma  linha em  cada  comando,
% dividindo como ficar melhor esteticamente).
% Remova  o  símbolo  de  comentário  (%)  de  \coorientador,  se  houver
% coorientador.
%  Em  \publicacao{011A/2019}:  o  número  final  será   fornecido   pela

\title{Coleta Simultânea de Eletroencefalograma \\ e Rastreamento Ocular: Ferramenta e Estudo de Caso}
\tituloficha{Coleta Simultânea de Eletroencefalograma e Rastreamento Ocular: Ferramenta e Estudo de Caso\\\phantom{}[Distrito Federal], 2019.} 
\titulocapaA{Coleta Simultânea de Eletroencefalograma e}
\titulocapaB{e Rastreamento Ocular: Ferramenta e Estudo de Caso}
\titulocapaC{}
\titulofichadois{Coleta Simultânea de Eletroencefalograma e Rastreamento Ocular: Ferramenta e Estudo de Caso}
\author{Ana Paula Sandes de Souza}
\nomeinvertido{Souza, Ana}
\orientador{Dr. Gerardo Antonio Idrobo Pizo}
%\coorientador{Nome do Coorientador}
\publicacao{011A/2022}
\data{Agosto de 2022}
\ano{2022}
\areaum{Neurociência Computacional} % Preencher com termos escolhidos para identificar a área
\areadois{Eletroencefalograma}
\areatres{Rastreamento Ocular}
\areaquatro{Sincronização de Sinais}
\endereco{anapaulasandes.s@gmail.com}
\cep{CEP 73105-904}

\membrobancainterno{DRª. Marília Miranda Forte Gomes}
\membrobancaexterno{Dr. Membro Externo}
%---}}}

% |--- Bibliotecas utilizadas ---|----------------------{{{
\usepackage[margin=1in]{geometry}
\usepackage{setspace}
\usepackage{multirow}
\usepackage{booktabs}
 \usepackage[brazil]{babel}
\usepackage{xfrac}
\usepackage{hyperref}
\hypersetup{
colorlinks = true,
linkcolor = black,
anchorcolor = blue,
citecolor = blue,
filecolor = blue,
urlcolor = blue
}
\usepackage{rotating}
\usepackage[margin=0.40in,font=small,labelfont=bf,labelsep=period]{caption}
%---}}}

% |- Formato de referências (use apenas uma das 2 linhas seguintes; comente a outra) -|-{{{
\newcommand{\formatobibliografia}{numero}
%\newcommand{\formatobibliografia}{autorano}

\ifthenelse{\equal{\formatobibliografia}{numero}}{
\bibliographystyle{plain}
}
{}

\ifthenelse{\equal{\formatobibliografia}{autorano}}{
\usepackage{apalike}
\bibliographystyle{apalike}
}
{}
%---}}}

% |--- Espaçamento, configuração de título de seções ---|----------------------{{{
\onehalfspacing

\makeatletter
\renewcommand{\section}{\@startsection
{section}
{1}
{0mm}
{-\baselineskip}
{0.5\baselineskip}
{\large\bfseries\scshape}}
\makeatother

\makeatletter
\renewcommand{\subsection}{\@startsection
{subsection}
{2}
{0mm}
{-\baselineskip}
{0.5\baselineskip}
{\bf\sffamily}}
\makeatother

\makeatletter
\renewcommand{\subsubsection}{\@startsection
{subsubsection}
{3}
{0mm}
{-\baselineskip}
{0.5\baselineskip}
{\bf\sffamily}}
\makeatother

\setlength{\parindent}{20pt}
\setlength{\parskip}{06pt}
\newcommand{\spaceinitialsname}{0.4mm}
\newcommand{\porcento}{\scalebox{0.5}{~}\scalebox{0.9}{\%}}
\newcommand{\scanner}{\emph{scanner}}
\newcommand{\scanners}{\emph{scanners}}
\newcommand{\cmcubico}{${\textrm{cm}^{\scalebox{0.7}{3} }}$}
\setcounter{secnumdepth}{3}
%\setcounter{tocdepth}{3}
%---}}}

% |--- Comandos especiais ---|----------------------{{{
\newcommand{\cmquad}{${\textrm{cm}^{\scalebox{0.7}{2}} }$}
\newcommand{\mmquad}{${\textrm{mm}^{\scalebox{0.7}{2}} }$}
\newcommand{\gcmquad}{${\textrm{g}}/{\textrm{cm}^{\scalebox{0.7}{2}} }$}
\newcommand{\subsecref}[1]{Seção~\ref{#1}}
\newcommand{\figref}[1]{Figura~\ref{#1}}
\newcommand{\etal}{\emph{et~al.}}
\newcommand{\Jawsonly}{{\emph{Jaws-Only}} }
\newcommand{\jawsonly}{{\emph{jaws-only}} }
\newcommand{\software}{\emph{software}}
\newcommand{\percentagesignscale}{0.8}
\newcommand{\percent}{\scalebox{\percentagesignscale}{~\%}}
\newcommand{\subsubsubsection}[1]{\vspace{16pt}\noindent\textbf{#1}\\[12pt]}
%---}}}

% |--- Diretório(s) com figuras (se desejar, inclua subdiretórios) ---|-------------{{{
\graphicspath{{figuras/}}
%---}}}

% |--- Lista de palavras que não podem ser separadas em sílabas ---|------------------{{{
\hyphenation{development results Commissioning possibility Philadelphia Devic Calculations Calculation Language}
%---}}}

% |--- Texto principal ---|----------------------{{{
\begin{document}

\maketitle

% Se desejar uma epígrafe, remova o % do início das próximas linhas (até ==============)
%\clearpage
%\hspace{1mm}
%
%\vfill
%
%\hspace{1mm}
%
%\begin{center}
%\emph{Epígrafe} \\
%Autor da epígrafe
%\end{center}
%
%\hspace{1mm}
%
%\vfill
%
%\hspace{1mm} 
% ==============

% Se desejar uma dedicatória, remova o % do início das próximas linhas (até ==============)
%\clearpage
%\hspace{1mm}
%
%\vfill
%
%\begin{flushright}
%\begin{itshape}
%Texto da dedicatória.
%\end{itshape}
%\end{flushright}
% ==============

% Se desejar incluir agradecimentos, remova o % do início das próximas linhas (até ==============)
% \clearpage
%\noindent{\bfseries{\maiusc{\large Agradecimentos}} }
%
%\vspace{24pt} Agradecimentos
%
%\noindent 
%\clearpage
% ==============

\newgeometry{bottom=0.8in, top=0.9in, left=0.9in, right=0.9in}

\noindent{\bfseries{\maiusc{\large Resumo}} }
\acresetall % Manter essa linha!
\vspace{12pt}

O eletroencefalograma (EEG) e o rastreamento ocular (ET) oferecem formas não-invasivas de se observar o comportamento do sistema nervoso e são importantes ferramentas na construção de bases de dados fisiológicos. O custo dos equipamentos de coleta e a sincronização dos dados coletados constituem gargalos na construção de datasets multimodais. Diferentes estudos relacionam a integração de dados fisiológicos a um maior poder classificatório em algoritmos de aprendizado supervisionado. O presente estudo observa os mais recentes métodos de sincronização de equipamentos de coleta e dados para a construção de datasets multimodas de EEG e ET, além de propor uma ferramenta de coleta simultânea de baixo custo que auxilie na construção de bases de dados fisiológicos. É esperado que uma forma acessível de construção de bases de dados multimodais incentive o desenvolvimento de novos algoritmos de aprendizado de máquina, e auxilie na criação de uma maior quantidade de datasets fisiológicos disponíveis para estudos futuros. 

\vspace{14pt}

\noindent{\textbf{Palavras-chave: }} EEG; ET; Sincronização; Base de Dados Fisiológicos;
\acresetall % Manter essa linha!
\clearpage
\restoregeometry
% \chapter{Abstract}
\noindent{\bfseries{\maiusc{\large Abstract}} }
\acresetall % Manter essa linha!
\vspace{24pt}

Electroencephalogram (EEG) and eye tracking (ET) are non-invasive ways of observing the nervous system behavior and are important tools in the construction of physiological databases. The equipment cost and synchronization of data are bottlenecks in the multimodal dataset construction. Studies relate physiological data integration to a higher classification accuracy in supervised learning algorithms. This study observes the latest methods of synchronizing data for building multimodal EEG and ET datasets trough the usage of commercially available equipment. It is expected that an affordable way of building multimodal databases will encourage the development of new machine learning algorithms, and increase the amount of physiological datasets available for future studies.

\vspace{14pt}

\noindent{\textbf{Keywords: }} EEG; ET; Synchronization; Physiological Dataset;
\acresetall % Manter essa linha!

\indice

\begin{center}

{\bfseries{\maiusc{\large Lista de Nomenclaturas e Abreviações}} }%
\end{center}

\acrodef{3DCRT}[3DCRT]{Radioterapia Conformacional 3D, do inglês \emph{3D Conformal Radiotherapy}}
\acrodef{AAPM}[AAPM]{Associação Americana de Física na Medicina, do inglês \emph{American Association of Physics in Medicine}}
\acrodef{CQ}[CQ]{Controle de Qualidade}
\acrodef{SPT}[SPT]{Sistema de Planejamento de Tratamento}

\begin{acronym}
\acro{3DCRT}{Radioterapia Conformacional 3D, do inglês \emph{3D Conformal Radiotherapy}}
\acro{AAPM}{Associação Americana de Física na Medicina, do inglês \emph{American Association of Physics in Medicine}}
\acro{CQ}{Controle de Qualidade}
\acro{SPT}{Sistema de Planejamento de Tratamento}
\end{acronym}

\clearpage

\pagenumbering{arabic}

\acresetall % Manter essa linha!

\chapter{Introdução}

Existe uma importante vantagem advinda do uso de bases fisiológicas chamadas multimodais, ou de mais de um tipo de dado fisiológico em algoritmos supervisionados: a possibilidade de conferir um maior poder classificatório em relação aos datasets unimodais (Kang et al., 2020; Thapaliya et al., 2019). Sobre os benefícios já alcançados com estes datasets, é possível citar: melhora de diagnóstico de transtornos neurológicos, como depressão e autismo (Kang et al., 2020; Thapaliya et al., 2019; Wu et al., 2021), maior poder de classificação de emoções (Guo et al., 2019; Zheng et al., 2019;  Lu et al, 2015; Zheng et al., 2014), e uma maior compreensão da ativação de mecanismos nervosos durante atividades de rotina, como leitura (Hollenstein et al., 2018). Um modelo específico de dataset fisiológico multimodal é constituído do eletroencefalograma (EEG) e rastreamento ocular (RO ou ET, da palavra em inglês Eye Tracking). Seu uso no treinamento de algoritmos classificatórios atestou sua aplicabilidade em diferentes contextos clínicos e acadêmicos, além de um aumento de acurácia na classificação de diferentes doenças nervosas e de emoções. 

%Se você deseja que o primeiro parágrafo de cada seção também tenha indentação, inclua no preâmbulo o comando \verb,\usepackage{indentfirst},.

\section{Contextualização de Problema}

Apesar das múltiplas vantagens, o acesso a estes datasets ainda é restrito. Sobre a coleta de EEG e ET, Kastrati et al. (2021) comenta:


\textit{“Coletar e classificar dados simultâneos de EEG e de rastreamento ocular é demorado e caro, pois requer equipamento e experiência para aquisição de EEG e rastreamento ocular. Portanto, o acesso a dados de EEG-ET gravados simultaneamente é altamente restrito, o que retarda significativamente o progresso neste campo”. – Kastrati et al. (2021).}

A redução do custo das coletas fisiológicas já vem sido abordada através de equipamentos comercialmente disponíveis. Um exemplo é o desenvolvimento de “smart watches”, pequenos computadores de pulso que permitem o acompanhamento da frequência cardíaca do usuário, além do monitoramento de outras atividades fisiológicas, como o sono. O Mindwave Mobile 2, do fabricante Neurosky®, é um exemplo de equipamento comercial que possibilita a captura de ondas cerebrais e métricas próprias do fabricante utilizadas para estimar medidas de atenção, meditação e a captura de piscadas dos usuários. Este equipamento permite o desenvolvimento de aplicações na forma de jogos interativos, neurofeedback e outras aplicações lúdicas (Neurosky). A respeito de seu uso em pesquisas científicas, ele já foi utilizado para estimar quais métodos de ensino despertavam maior atenção em alunos do ensino fundamental, estimar personalidade de participantes através da apresentação de vídeo clips eleitos para instigar um determinado traço de personalidade, e classificação de emoções. Além deste equipamento para coleta de EEG, também existem equipamentos disponíveis comercialmente para a coleta de ET, como o GP3 (Gazepoint®). 


Com o propósito de aumentar a acessibilidade aos datasets multimodais e suas amplas vantagens de uso, o presente projeto tem por objetivo a criação de uma ferramenta de coleta simultânea de EEG e RO acessível a partir da coleta de dados de dois equipamentos comerciais – GP3 para a coleta de RO, e Mindwave Mobile 2 para a coleta de dados de ativação neuronal. A ferramenta terá como output um dataset constituído de dados de EEG e RO coletados simultaneamente. O output será testado em um estudo de caso através da análise de performance de quatro diferentes algoritmos classificatórios treinados com o output para classificar entre duas possíveis atividades. 


\section{Objetivos}

Criar uma ferramenta capaz de gerar um dataset de EEG e ET coletado de forma síncrona e validar o dataset através da performance de algoritmos de aprendizado supervisionado treinados com ele.

\subsection{Objetivos Específicos}

\begin{enumerate}
    \item Criar código para coleta simultânea de EEG e RO;
    \item Criar dataset multimodal a partir da fusão de dados de EEG e RO coletados pela ferramenta e classificado de acordo com o estímulo apresentado ao participante;
    \item Treinar diferentes algoritmos supervisionados com o dataset multimodal gerado;
\end{enumerate}

\section{Justificativa}
Apesar da existência de técnicas que permitam a extração de mais de um modo de dados fisiológicos a partir de um equipamento apenas - como a extração da posição ocular a partir de assinaturas elétricas em dados de EEG, estes métodos necessitam de um grande volume de dados, o que exige equipamentos mais refinados e, por vezes, uma grande disponibilidade de tempo para criação dos datasets e deslocamento de participantes até a estação de coleta. A coleta de ET e EEG por equipamento comercial e acessível, seria, portanto, uma alternativa que permite um maior controle no desenvolvimento de estudos com algoritmos de aprendizado de máquina, sem depender de equipamentos de alto custo ou deslocamento de participantes até a estação. 


É argumentado que um maior acesso a construção de bases de dados multimodais poderá expandir e aprofundar os avanços em neurociência e estudos comportamentais, ao proporcionar um maior controle do design de experimento e expandir a quantidade de datasets fisiológicos gerados. 

\section{Organização do Documento}
O presente texto tem nove capítulos. O primeiro capítulo trata da contextualização do problema, objetivos gerais e específicos, e a justificativa para a abordagem selecionada. 

O segundo capítulo trata do referencial teórico, levantando pontos históricos importantes ao desenvolvimento desta pesquisa, uma introdução ao que seriam os sinais capturados pelos dois equipamentos de EEG e ET, características dos diferentes tipos de equipamento de captura e aborda a conversão de sinais analógicos para digital. 

O terceiro capítulo trata do processo da aquisição e do tratamento de sinais fisiológicos a serem classificados por algoritmos de aprendizado de máquina supervisionado – processos como remoção de ruído e seleção de características. Também aborda métodos de fusão de bases de dados de diferentes fontes - a nível de característica e a nível de decisão. 

O quarto capítulo apresenta os métodos de avaliação dos algoritmos classificatórios. 

O quinto capítulo aprofunda nos possíveis métodos de sincronização de coleta e de bases de dados.

O sexto capítulo introduz a proposta de ferramenta de coleta simultânea elaborada por código e aborda o método de sincronização off-line, com sincronização por código temporal. 

O sétimo capítulo aborda o cronograma.

O oitavo capítulo aborda os resultados esperados do estudo.


O nono capítulo apresenta as referencias do estudo.

\chapter{Séries Temporais}

Séries temporais são sequencias de pontos ao longo do tempo, onde pontos vizinhos são dependentes (Ehlers, 2007). 
As séries podem ser contínuas ou discretas. Um exemplo de sinal contínuo é a diferença da voltagem de neurônios
capturada por eletrodos ao longo do tempo. 

\chapter{Sincronização}

O uso de equipamentos com função exclusiva de sincronização para coletas simultâneas é comum em pesquisas ambientes academicos e clínicos.
A proposta de oferecer maior acessibilidade através da redução de custo e desenvolvimento de novas tecnologias encontra, portanto, um desafio a respeito 
de como realizar a sincronização dos dados fisiológicos sem abrir mão da praticidade e custo dos equiapementos desenvolvidos. 
Algumas propostas já foram exploradas a respeito, como o uso de piscadas e código temporal para garantir a sincronização de EEG e ET (Bækgaard et al. 2015, Notaro et al. 2018).

\section{Frequência de Coleta}
Como os sinais análogos são convertidos para sinais digitais, existe uma perda de informação por esta conversão. 
A \textbf{resolução de frequência} mede o espaço entre duas frequências. 

$$srate/N$$

Srate = sampling rate 
N = Número de amostras

\subsection{Frequência Nyquist}
É a frequência mais rápida onde o sinal pode ser medido, onde é estabelecido que a maior frequência que podemos medir é a metade 
da frequência de coleta.


\section{Sincronização com Timecode}
Notaro et al. (2018) faz uso do código temporal, ou \textit{timecode}, para sincronizar dados de EEG, ET e dados comportamentais 
coletados de participantes enquanto estes faziam atividades de um site de aprendizagem de linguas. O driver
do fabricante do equipamento comercial de EEG utilizado permite alteração da latência da coleta de dados, que
foi modificada do valor padrão de 16 milissegundos para 1 millisegundo, afim de aumentar a precisão do equipamento.
A informação da ocorrência de clicks no site foi retina na forma de milissegundos (HH:MM:SS:MsMsMs), e esta informação foi utilizada 
para sincronizar dados de ET, EEG e movimentação de mouse. 

\section{Sincronização com Piscadas}
Piscadas duram cerca de 200 milissegundos em média e podem indicar estados de alerta (Caffier, 2013). Piscadas também aparecem 
em dados de EEG de forma característica, podendo alcançar uma amplitude de sinal acima de 200 microvolts em eletrodos próximos a órbita ocular (Hoffmann e Falkenstein, 2008). Assim sendo, é possível realizar uma sincronização por piscadas ao se detectar 
o movimento em ambos os equiapmentos de coleta. No caso do EEG, as piscadas são comumente descartadas como artefatos indesejáveis. Já no estudo de 
Bækgaard et al. (2015), elas são a assinatura de sincronização entre os equipamentos de coleta de EEG e ET em função de sua onda característica (geralmente muitos milivolts acima do sinal do EEG), e de também 
ser detectdo através dos equipamentos de rastreamento ocular.

\subsection{Identificação no Sinal do EEG}
Para se detectar a piscada através de um sinal, é possível tentar realizar o método de \textit{Independent Component Analysis}, ou análise de 
componente independente, mas as características do sinal de piscada também permite outras abordagens, como a identificação por função de probabilidade.
Considerando o movimento de maior característica da piscada, é preferível se calcular a probabilidade do movimento de fechar os olhos 
ao movimento de abertura, em função de uma variação em tempo ser mais comumente encontrada na fase de abertura (Caffier, 2013).

\subsection{Identificação no Sinal de ET}
Como o equipamento de rastreamento procura encontrar sinais da movimentação ocular, ele também detecta a ausencia desse sinal. No estudo 
de Bækgaard et al. (2015), uma perda de até 500 milissegundos foi considerada como indicador da ocorrência de uma piscada. 

\subsection{Correlação Cruzada}


\section{Códigos para Sincronização}
Alguns equipamentos podem se beneficiar da existencia de \textit{toolboxes} ou bibliotecas direcionadas à sincronização. É o caso 
dos equipamentos Tobbii na solução de EEG-Eye para a linguagem MATLAB. Uma forma de se fazer sincronização é atrav´s 


\chapter{Fundamentação Teórica}\label{chap:FT}

Este capítulo pode ter outro nome, e na verdade sugiro um nome mais específico (indique no título sobre o que trata a fundamentação em questão).

Inclua as seções que se façam necessárias.

\section{Observações Sobre Figuras}

Cada figura deve ser citada ao menos uma vez antes de aparecer no texto. As figuras devem ser numeradas no formato x.y, com x o número do capítulo e y o número da figura dentro do capítulo, e devem incluir uma legenda com o número e com um texto explicativo abaixo da figura em si. Como um exemplo, a Figura~\ref{fig:acelerador} ilustra um acelerador linear do Hospital Universitário de Brasília. Note que a citação foi com a palavra ``figura'' em letra maiúscula, como aparece na legenda. Note ainda que a legenda é em fonte menor do que o texto principal, e com margem reduzida em relação ao resto do texto.

\begin{figure}[h]
\centering
\includegraphics[width=100mm]{Acelerador2}
\caption[Exemplo de um acelerador linear utilizado no Hospital Universitário de Brasília.]{Exemplo de um acelerador linear utilizado no Hospital Universitário de Brasília. Os ângulos de 360${^{\textrm{o}} }$, 180${^{\textrm{o}} }$ e 180${^{\textrm{o}} }$ indicam os possíveis valores de rotação do acelerador e do \emph{gantry}. Os valores em centímetros indicam as dimensões do \emph{gantry} e as distâncias em relação à mesa e ao chão. Fonte:~\cite{Avelino2013}.}\label{fig:acelerador}
\end{figure}

Se você inserir figuras de outras fontes (livros, artigos, etc), deve incluir a fonte na legenda. Diga explicitamente ``Fonte: [X]'', sendo X a referência de onde foi tirada a figura. Ou use ``Adaptada de [X]'', caso a figura tenha sido modificada (por exemplo, traduzida). Não abuse, no entanto, da utilização de figuras de outras fontes. Dê preferência a trabalhos de sua autoria. Note que uma figura de outra fonte, mesmo com a devida citação, só poderia ser utilizada com autorização por escrito, para evitar processo por direitos autorais. Já o caso de inclusão de figuras de outras fontes sem a devida citação constitui plágio, sendo o autor do plágio sujeito à perda do título eventualmente obtido com a publicação e de outros direitos dela decorrentes.

\textbf{No caso de figuras de sua própria autoria, não indique isso na legenda. Não escreva, por exemplo, ``Fonte: o autor''}. Já se assume no texto que todo o material apresentado é produção do autor indicado, e os outros casos, que devem ser comparativamente poucos, é que devem ser explicitados.

\section{Observações Sobre Equações}

As equações são normalmente escritas de forma centralizada ao longo da direção horizontal, e com uma numeração à direita no caso das equações que são citadas. As equações que não são citadas posteriormente não precisam ser numeradas. Quando há numeração, ela aparece entre parênteses, e no formado x.y, com x o número do capítulo e y o número da equação dentro do capítulo.

Segue um exemplo de uma equação. Em um triângulo retângulo, a medida da hipotenusa é dada por
\begin{equation}\label{eq:hipotenusa}
a = \sqrt{b^2 + c^2},
\end{equation}
com ${b}$ e ${c}$ as medidas dos dois catetos.

Note em \eqref{eq:hipotenusa} que a equação faz parte do texto, no sentido de que ela não interrompe o fluxo da frase iniciada por ``Em um triângulo''. Não se deve, por exemplo, escrever ``a medida da hipotenusa é dada pela Equação 2.1'', e então colocar a equação abaixo como se fosse um objeto à parte do parágrafo (como acontece com as figuras e tabelas -- estas sim não se inserem no próprio texto, e são referenciadas como objetos independentes do parágrafo).

Por este motivo, as equações devem ser pontuadas conforme o texto normal. Elas devem ser seguidas, por exemplo, de ponto, vírgula, ou ponto-e-vírgula, conforme o fluxo do texto, a não ser que o texto imediatamente continue com a palavra ``e''.

Além disso, observe que todos os termos de uma equação que não foram previamente definidos devem ser definidos logo em seguida, como no caso de \eqref{eq:hipotenusa}. Os termos ${b}$ e ${c}$ foram definidos imediatamente após a equação. Nunca deve haver termos numa equação que não são explicitamente definidos no texto.

Segue um outro exemplo. A transformada discreta de Fourier de um sinal ${x}$ de comprimento ${N}$ é dada por
\begin{equation}
\hat{x}[k]=\sum_{n=0}^{N-1}x[n]\exp\left(-j\frac{2\pi nk}{N}\right),
\end{equation}
sendo ${j}$ a unidade imaginária e ${k}$ o índice de frequência considerado, com ${k\in\left\{0,1,\ldots,N\right\}}$.

\chapter{Materiais e Métodos}\label{chap:Metodologia}

Este capítulo pode ter outro nome, e na verdade sugiro um nome mais específico; indique no título sobre os tópicos metodológicos tratados. Pode ser usado mais de um capítulo para esses tópicos, se necessário.

Inclua as seções que se façam necessárias.

\section{Dicas para o capítulo}
Dicas importantes que devem ser contempladas neste capítulo, segundo~\cite{marconi.lakatos:2003}:
\begin{itemize}
\item Verificar se o capítulo responde as seguintes questões: Como? Com quê? Onde? Quanto?
\item A linguagem do projeto deve ser escrita com tempo verbal no futuro e da dissertação no passado.
\item É importante mencionar sobre: tipo de pesquisa (bibliográfica, descritiva, documental, experimental etc), dados (fonte de dados, forma de obtenção), população e amostra, tratamento e análise dos dados (descrição mais detalhada do método -- ou métodos -- que serão utilizados), limitações da pesquisa.
\end{itemize}

\section{Observações Sobre Quadros e Tabelas}

Quadros e tabelas são de uso semelhante às figuras, no que diz respeito à numeração, uso de legenda, e necessidade de citar ao menos uma vez antes da ocorrência. No entanto, no caso dos quadros e tabelas a legenda deve ser colocada acima, e não abaixo como nas figuras.

A Tabela~\ref{tab:exemplo} ilustra esse uso. Observe que a citação de uma tabela específica (pelo número) é com a palavra ``tabela'' em maiúscula, ao contrário da referência a tabelas em geral. Note que em uma tabela as bordas são horizontais (não use bordas verticais para separar colunas), e não são necessárias bordas para separar cada linha. Separe apenas as linhas do início, fim, e dos indicadores dos campos presentes, como no exemplo. Podem ser usadas bordas horizontais para separar regiões distintas de dados (seções de dados), se necessário.

\begin{table}[h]
\centering
\caption{Parâmetros utilizados na implementação do método de deteção de bordas proposto, em cada configuração considerada.}\label{tab:exemplo}
\begin{tabular}{ccllll}
\cline{1-5}
\multirow{2}{*}{Configuração} && \multicolumn{3}{l}{\hspace*{12pt}Parâmetro}&  \\
&& \hspace{4pt}A\hspace{4pt} & \hspace{4pt}B\hspace{4pt} & \hspace{4pt}C\hspace{4pt} & \\ \cline{1-5}
1                             && \hspace{4pt}10\hspace{4pt}        & \hspace{4pt}5\hspace{4pt}       & \hspace{4pt}2\hspace{4pt}       &  \\
2                             && \hspace{4pt}20\hspace{4pt}        & \hspace{4pt}5\hspace{4pt}       & \hspace{4pt}3\hspace{4pt}       &  \\
3                             && \hspace{4pt}30\hspace{4pt}        & \hspace{4pt}8\hspace{4pt}       & \hspace{4pt}5\hspace{4pt}       &  \\ \cline{1-5}
\end{tabular}
\end{table}

O Quadro~\ref{quadro:exemplo} é um outro exemplo. Note que um quadro se diferencia de uma tabela pelo uso de campos fechados, por meio de linhas horizontais e verticais. As tabelas são mais usadas para dados quantitativos, enquanto quadrados são mais usados quando há descrições textuais (mesmo que haja dados quantitativos também).

\begin{quadro}
\caption{Exemplo de um quadro (retirado de~\cite{Gomes2011}): \emph{Variáveis explicativas que representam características socioeconômicas dos idosos.} Fonte:~\cite{Gomes2011}}\label{quadro:exemplo}
\begin{center}
\scalefont{0.705}
\begin{tabular}{|l|l|l|}
\hline
\hfill Variável\hfill\hspace{1mm} & \hfill Descrição${^{*}}$\hfill\hspace{1mm} & \hfill Categorização\hfill\hspace{1mm}\\
\hline
Nível de escolaridade & Número de anos de estudo (A5a, A5b, A6) & \begin{tabular}{l}Nenhum\\1 a 7 anos\\8 anos e mais\end{tabular}\\
\hline
Tem seguro/plano privado de saúde?&\hspace{-06pt}\begin{tabular}{l}Que tipo de seguro de saúde o(a) Sr.(a)\\ tem? (F1)\end{tabular} & \begin{tabular}{l}Sim\\Não\end{tabular}\\
\hline
Tem casa própria?&Esta casa é: (J2) & \begin{tabular}{l}Sim\\Não\end{tabular}\\
\hline
Uso de serviços de saúde&\hspace{-06pt}\begin{tabular}{l}Durante os últimos 12 meses, aonde o(a)\\ Sr.(a) foi quando se sentiu doente ou quando\\ precisou fazer uma consulta de saúde? (F3)\end{tabular} & \begin{tabular}{l}Usou\\Não usou\end{tabular}\\
\hline
Estado nutricional&\hspace{-06pt}\begin{tabular}{l}Com relação a seu estado nutricional o(a) \\Sr.(a) se considera bem nutrido? (C22i)\end{tabular} & \begin{tabular}{l}Bem nutrido\\Não está bem nutrido\end{tabular}\\
\hline
\end{tabular}
\scalefont{1.4184}
\end{center}
\vspace{-12pt}
Fonte: Estudo SABE.\\
\emph{${^{\textrm{*} }}$Os códigos em parênteses na descrição das variáveis se referem à identificação da variável no banco de dados do Estudo SABE.}~\cite{Gomes2011}
\end{quadro}

\chapter{Resultados e Discussões}\label{chap:RD}

\begin{table}[h]
\centering
\caption{Fatores de qualidades medidos em função do número de amostras, nos testes de reconstrução realizados.}\label{tab:qualidade}
\begin{tabular}{cc}
\toprule
Número de amostras & Fator de qualidade\\
\midrule
10 & 0.30\\
20 & 0.45\\
30 & 0.60\\
40 & 0.90\\
50 & 0.93\\
\bottomrule
\end{tabular}
\end{table}

\begin{table}[h]
\centering
\caption{Outro exemplo de tabela.}\label{tab:outroexemplo}
\begin{tabular}{ccccc}
    \toprule
    a     & b     & c     & d     & e \\
    \midrule
    10    & 20    & 30    & 40    & 50 \\
    100   & 200   & 300   & 400   & 500 \\
    \bottomrule
    \end{tabular}%
\end{table}

\chapter{Conclusão}\label{chap:Conclusao}

\renewcommand\bibname{\Large\scshape Lista de Referências}
\addcontentsline{toc}{chapter}{\bf Lista de Referências}
\bibliography{referencias}

% |--- Exemplos de Apêndices ---|----------------------{{{
% Início do Apêndice
\newcounter{apendice}
\counterwithin{figure}{apendice}
\counterwithin{table}{apendice}
\renewcommand{\theapendice}{\Alph{apendice}}
\DeclareRobustCommand{\novoapendice}[1]{%
    \refstepcounter{apendice}%
    \phantom{\theapendice}\label{#1}}

% Exemplo 1
\clearpage
\begin{flushright}
\novoapendice{apendice_exemplo}
\scalebox{1.3}{\bfseries\scshape Apêndice~\ref{apendice_exemplo}}
\addcontentsline{toc}{chapter}{Apêndice~\ref{apendice_exemplo}}
\end{flushright}

\noindent\begin{large}{\bfseries\scshape Exemplo de Apêndice}\end{large} \label{sec:apendice1}

\vspace{24pt}

Se você desejar, pode incluir ao final um ou mais apêndices, e um ou mais anexos. Caso não queira, é só remover todo o conteúdo começando na linha marcada por ``\%~Início~do~Apêndice'', até a linha anterior a ``\verb|\end{document}|''.

% Exemplo 2
\clearpage
\begin{flushright}
\novoapendice{apendice_outro_exemplo}
\scalebox{1.3}{\bfseries\scshape Apêndice~\ref{apendice_outro_exemplo}}
\addcontentsline{toc}{chapter}{Apêndice~\ref{apendice_outro_exemplo}}
\end{flushright}

\noindent\begin{large}{\bfseries\scshape Outro Exemplo de Apêndice}\end{large} \label{sec:apendice2}

\vspace{24pt}

Se você desejar, pode incluir ao final um ou mais apêndices, e um ou mais anexos. Caso não queira, é só remover todo o conteúdo começando na linha marcada por ``\%~Início~do~Apêndice'', até a linha anterior a ``\verb|\end{document}|''.
%---}}}

% |--- Exemplos de Anexos ---|----------------------{{{
% Início do anexos
\newcounter{anexo}
\counterwithin{figure}{anexo}
\counterwithin{table}{anexo}
\renewcommand{\theanexo}{\Alph{anexo}}
\DeclareRobustCommand{\novoanexo}[1]{%
    \refstepcounter{anexo}%
    \phantom{\theanexo}\label{#1}}

% Exemplo 1
\clearpage
\begin{flushright}
\novoanexo{anexo_exemplo}
\scalebox{1.3}{\bfseries\scshape Anexo~\ref{anexo_exemplo}}
\addcontentsline{toc}{chapter}{Anexo~\ref{anexo_exemplo}}
\end{flushright}

\noindent\begin{large}{\bfseries\scshape Exemplo de Anexo}\end{large} \label{sec:anexo1}

\vspace{24pt}

Se você desejar, pode incluir ao final um ou mais apêndices, e um ou mais anexos. Caso não queira, é só remover todo o conteúdo começando na linha marcada por ``\%~Início~do~Apêndice'', até a linha anterior a ``\verb|\end{document}|''.

% Exemplo 2
\clearpage
\begin{flushright}
\novoanexo{anexo_outro_exemplo}
\scalebox{1.3}{\bfseries\scshape Anexo~\ref{anexo_outro_exemplo}}
\addcontentsline{toc}{chapter}{Anexo~\ref{anexo_outro_exemplo}}
\end{flushright}

\noindent\begin{large}{\bfseries\scshape Outro Exemplo de Anexo}\end{large} \label{sec:anexo2}

\vspace{24pt}

Se você desejar, pode incluir ao final um ou mais apêndices, e um ou mais anexos. Caso não queira, é só remover todo o conteúdo começando na linha marcada por ``\%~Início~do~Apêndice'', até a linha anterior a ``\verb|\end{document}|''.
%---}}}

\end{document} 
%---}}}
