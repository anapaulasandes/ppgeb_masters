\documentclass[a4paper, 12pt]{ppgeb}

% |--- Títulos, autor, banca ---|----------------------{{{
% Autor:
% Substituta  as informações nos comandos a seguir, até a linha começando
% com \membrobancaexterno.
% Em \title: título na forma principal, como aparecerá em algumas páginas
% Em \tituloficha: título como aparecerá na ficha catalográfica; idêntico
% ao anterior, mas com possíveis quebras manuais de linha (usar \\ quando
% necessário, para ajustar as mudanças de linha na ficha catalográfica).
% Em   \titulocapaA,   \titulocapaB,   \titulocapaC:  título para a capa,
% dividido em no  máximo 3  linhas (coloque uma  linha em  cada  comando,
% dividindo como ficar melhor esteticamente).
% Remova  o  símbolo  de  comentário  (%)  de  \coorientador,  se  houver
% coorientador.
%  Em  \publicacao{011A/2019}:  o  número  final  será   fornecido   pela

\title{Coleta Simultânea de Eletroencefalograma \\ e Rastreamento Ocular: Ferramenta e Estudo de Caso}
\tituloficha{Coleta Simultânea de Eletroencefalograma e Rastreamento Ocular: Ferramenta e Estudo de Caso\\\phantom{}[Distrito Federal], 2019.} 
\titulocapaA{Coleta Simultânea de Eletroencefalograma e}
\titulocapaB{e Rastreamento Ocular: Ferramenta e Estudo de Caso}
\titulocapaC{}
\titulofichadois{Coleta Simultânea de Eletroencefalograma e Rastreamento Ocular: Ferramenta e Estudo de Caso}
\author{Ana Paula Sandes de Souza}
\nomeinvertido{Souza, Ana}
\orientador{Dr. Gerardo Antonio Idrobo Pizo}
%\coorientador{Nome do Coorientador}
\publicacao{011A/2022}
\data{Agosto de 2022}
\ano{2022}
\areaum{Neurociência Computacional} % Preencher com termos escolhidos para identificar a área
\areadois{Eletroencefalograma}
\areatres{Rastreamento Ocular}
\areaquatro{Sincronização de Sinais}
\endereco{anapaulasandes.s@gmail.com}
\cep{CEP 73105-904}

\membrobancainterno{DRª. Marília Miranda Forte Gomes}
\membrobancaexterno{Dr. Membro Externo}
%---}}}

% |--- Bibliotecas utilizadas ---|----------------------{{{
\usepackage[margin=1in]{geometry}
\usepackage{setspace}
\usepackage{multirow}
\usepackage{booktabs}
 \usepackage[brazil]{babel}
\usepackage{xfrac}
\usepackage{hyperref}
\hypersetup{
colorlinks = true,
linkcolor = black,
anchorcolor = blue,
citecolor = blue,
filecolor = blue,
urlcolor = blue
}
\usepackage{rotating}
\usepackage[margin=0.40in,font=small,labelfont=bf,labelsep=period]{caption}
%---}}}

% |- Formato de referências (use apenas uma das 2 linhas seguintes; comente a outra) -|-{{{
\newcommand{\formatobibliografia}{numero}
%\newcommand{\formatobibliografia}{autorano}

\ifthenelse{\equal{\formatobibliografia}{numero}}{
\bibliographystyle{plain}
}
{}

\ifthenelse{\equal{\formatobibliografia}{autorano}}{
\usepackage{apalike}
\bibliographystyle{apalike}
}
{}
%---}}}

% |--- Espaçamento, configuração de título de seções ---|----------------------{{{
\onehalfspacing

\makeatletter
\renewcommand{\section}{\@startsection
{section}
{1}
{0mm}
{-\baselineskip}
{0.5\baselineskip}
{\large\bfseries\scshape}}
\makeatother

\makeatletter
\renewcommand{\subsection}{\@startsection
{subsection}
{2}
{0mm}
{-\baselineskip}
{0.5\baselineskip}
{\bf\sffamily}}
\makeatother

\makeatletter
\renewcommand{\subsubsection}{\@startsection
{subsubsection}
{3}
{0mm}
{-\baselineskip}
{0.5\baselineskip}
{\bf\sffamily}}
\makeatother

\setlength{\parindent}{20pt}
\setlength{\parskip}{06pt}
\newcommand{\spaceinitialsname}{0.4mm}
\newcommand{\porcento}{\scalebox{0.5}{~}\scalebox{0.9}{\%}}
\newcommand{\scanner}{\emph{scanner}}
\newcommand{\scanners}{\emph{scanners}}
\newcommand{\cmcubico}{${\textrm{cm}^{\scalebox{0.7}{3} }}$}
\setcounter{secnumdepth}{3}
%\setcounter{tocdepth}{3}
%---}}}

% |--- Comandos especiais ---|----------------------{{{
\newcommand{\cmquad}{${\textrm{cm}^{\scalebox{0.7}{2}} }$}
\newcommand{\mmquad}{${\textrm{mm}^{\scalebox{0.7}{2}} }$}
\newcommand{\gcmquad}{${\textrm{g}}/{\textrm{cm}^{\scalebox{0.7}{2}} }$}
\newcommand{\subsecref}[1]{Seção~\ref{#1}}
\newcommand{\figref}[1]{Figura~\ref{#1}}
\newcommand{\etal}{\emph{et~al.}}
\newcommand{\Jawsonly}{{\emph{Jaws-Only}} }
\newcommand{\jawsonly}{{\emph{jaws-only}} }
\newcommand{\software}{\emph{software}}
\newcommand{\percentagesignscale}{0.8}
\newcommand{\percent}{\scalebox{\percentagesignscale}{~\%}}
\newcommand{\subsubsubsection}[1]{\vspace{16pt}\noindent\textbf{#1}\\[12pt]}
%---}}}

% |--- Diretório(s) com figuras (se desejar, inclua subdiretórios) ---|-------------{{{
\graphicspath{{figuras/}}
%---}}}

% |--- Lista de palavras que não podem ser separadas em sílabas ---|------------------{{{
\hyphenation{development results Commissioning possibility Philadelphia Devic Calculations Calculation Language}
%---}}}

% |--- Texto principal ---|----------------------{{{
\begin{document}

\maketitle

% Se desejar uma epígrafe, remova o % do início das próximas linhas (até ==============)
%\clearpage
%\hspace{1mm}
%
%\vfill
%
%\hspace{1mm}
%
%\begin{center}
%\emph{Epígrafe} \\
%Autor da epígrafe
%\end{center}
%
%\hspace{1mm}
%
%\vfill
%
%\hspace{1mm} 
% ==============

% Se desejar uma dedicatória, remova o % do início das próximas linhas (até ==============)
%\clearpage
%\hspace{1mm}
%
%\vfill
%
%\begin{flushright}
%\begin{itshape}
%Texto da dedicatória.
%\end{itshape}
%\end{flushright}
% ==============

% Se desejar incluir agradecimentos, remova o % do início das próximas linhas (até ==============)
% \clearpage
%\noindent{\bfseries{\maiusc{\large Agradecimentos}} }
%
%\vspace{24pt} Agradecimentos
%
%\noindent 
%\clearpage
% ==============

\newgeometry{bottom=0.8in, top=0.9in, left=0.9in, right=0.9in}

\noindent{\bfseries{\maiusc{\large Resumo}} }
\acresetall % Manter essa linha!
\vspace{12pt}

O eletroencefalograma (EEG) e o rastreamento ocular (ET - \textit{Eye Tracking}) são 
formas não-invasivas de se observar o comportamento do sistema nervoso através da coleta da atividade
neural e posicionamento ocular ao longo do tempo. Desta forma, são importantes ferramentas na construção de bases de dados fisiológicos. 
Bases de dados com mais de um tipo de dado fisiológico conferem vantagens a respeito
do uso em algoritmos classificatórios, apresentando uma maior acurácia do que algortimos treinados
com datasets unimodais.
Apesar das vantagens de uso, o acesso a estes datasets ainda é restrito devido ao custo
dos equipamentos e a complexidade da sincronização entre diferentes sensores.
O presente estudo tem por objetivo apresentar uma ferramenta construída a partir de equipamentos
comerciais que tem como \textit{output} um dataset multimodal de EEG e ET coletados simultaneamente.
É esperado que a ferramenta promova acessibilidade a datasets multimodais ao gerar uma coleta com 
equipamentos de baixo custo (relativo a equipamentos clínicos) e incentive o desenvolvimento de diferentes áreas de pesquisa que possam se beneficiar do acesso facilitado da coleta de dados fisiológicos. A ferramenta será 
avaliada a respeito de sua capacidade de sincronização de EEG e ET, e o dataset multimodal será utilizado para treinar algoritmos 
classificatórios no software Orange em um estudo de caso com apresentação de estímulos emocionais durante o momento de coleta.
A performance dos algortimos será utilizada para argumentar sobre a possível aplicabilidade da ferramenta em estudos futuros.

\vspace{14pt}

\noindent{\textbf{Palavras-chave: }} EEG; ET; Sincronização; Base de Dados Fisiológicos;
\acresetall % Manter essa linha!
\clearpage
\restoregeometry
% \chapter{Abstract}
\noindent{\bfseries{\maiusc{\large Abstract}} }
\acresetall % Manter essa linha!
\vspace{24pt}

Electroencephalogram (EEG) and eye tracking (ET) are non-invasive ways of observing the nervous system behavior and are important tools in the construction of physiological databases. The equipment cost and synchronization of data are bottlenecks in the multimodal dataset construction. Studies relate physiological data integration to a higher classification accuracy in supervised learning algorithms. This study observes the latest methods of synchronizing data for building multimodal EEG and ET datasets trough the usage of commercially available equipment. It is expected that an affordable way of building multimodal databases will encourage the development of new machine learning algorithms, and increase the amount of physiological datasets available for future studies.

\vspace{14pt}

\noindent{\textbf{Keywords: }} EEG; ET; Synchronization; Physiological Dataset;
\acresetall % Manter essa linha!

\indice

\begin{center}

{\bfseries{\maiusc{\large Lista de Nomenclaturas e Abreviações}} }%
\end{center}

\acrodef{3DCRT}[3DCRT]{Radioterapia Conformacional 3D, do inglês \emph{3D Conformal Radiotherapy}}
\acrodef{AAPM}[AAPM]{Associação Americana de Física na Medicina, do inglês \emph{American Association of Physics in Medicine}}
\acrodef{CQ}[CQ]{Controle de Qualidade}
\acrodef{SPT}[SPT]{Sistema de Planejamento de Tratamento}

\begin{acronym}
\acro{3DCRT}{Radioterapia Conformacional 3D, do inglês \emph{3D Conformal Radiotherapy}}
\acro{AAPM}{Associação Americana de Física na Medicina, do inglês \emph{American Association of Physics in Medicine}}
\acro{CQ}{Controle de Qualidade}
\acro{SPT}{Sistema de Planejamento de Tratamento}
\end{acronym}

\clearpage

\pagenumbering{arabic}

\acresetall % Manter essa linha!

\chapter{Introdução}

Existe uma importante vantagem advinda do uso de bases fisiológicas chamadas multimodais, ou de mais de um tipo de dado fisiológico em algoritmos supervisionados: a possibilidade de conferir um maior poder classificatório em relação aos datasets unimodais (Kang et al., 2020; Thapaliya et al., 2019). Sobre os benefícios já alcançados com estes datasets, é possível citar: melhora de diagnóstico de transtornos neurológicos, como depressão e autismo (Kang et al., 2020; Thapaliya et al., 2019; Wu et al., 2021), maior poder de classificação de emoções (Guo et al., 2019; Zheng et al., 2019;  Lu et al, 2015; Zheng et al., 2014), e uma maior compreensão da ativação de mecanismos nervosos durante atividades de rotina, como leitura (Hollenstein et al., 2018). Um modelo específico de dataset fisiológico multimodal é constituído do eletroencefalograma (EEG) e rastreamento ocular (RO ou ET, da palavra em inglês Eye Tracking). Seu uso no treinamento de algoritmos classificatórios atestou sua aplicabilidade em diferentes contextos clínicos e acadêmicos, além de um aumento de acurácia na classificação de diferentes doenças nervosas e de emoções. 

%Se você deseja que o primeiro parágrafo de cada seção também tenha indentação, inclua no preâmbulo o comando \verb,\usepackage{indentfirst},.

\section{Contextualização de Problema}

Apesar das múltiplas vantagens, o acesso a estes datasets ainda é restrito. Sobre a coleta de EEG e ET, Kastrati et al. (2021) comenta:


\textit{“Coletar e classificar dados simultâneos de EEG e de rastreamento ocular é demorado e caro, pois requer equipamento e experiência para aquisição de EEG e rastreamento ocular. Portanto, o acesso a dados de EEG-ET gravados simultaneamente é altamente restrito, o que retarda significativamente o progresso neste campo”. – Kastrati et al. (2021).}

A redução do custo das coletas fisiológicas já vem sido abordada através de equipamentos comercialmente disponíveis. Um exemplo é o desenvolvimento de “smart watches”, pequenos computadores de pulso que permitem o acompanhamento da frequência cardíaca do usuário, além do monitoramento de outras atividades fisiológicas, como o sono. O Mindwave Mobile 2, do fabricante Neurosky®, é um exemplo de equipamento comercial que possibilita a captura de ondas cerebrais e métricas próprias do fabricante utilizadas para estimar medidas de atenção, meditação e a captura de piscadas dos usuários. Este equipamento permite o desenvolvimento de aplicações na forma de jogos interativos, neurofeedback e outras aplicações lúdicas (Neurosky). A respeito de seu uso em pesquisas científicas, ele já foi utilizado para estimar quais métodos de ensino despertavam maior atenção em alunos do ensino fundamental, estimar personalidade de participantes através da apresentação de vídeo clips eleitos para instigar um determinado traço de personalidade, e classificação de emoções. Além deste equipamento para coleta de EEG, também existem equipamentos disponíveis comercialmente para a coleta de ET, como o GP3 (Gazepoint®). 


Com o propósito de aumentar a acessibilidade aos datasets multimodais e suas amplas vantagens de uso, o presente projeto tem por objetivo a criação de uma ferramenta de coleta simultânea de EEG e RO acessível a partir da coleta de dados de dois equipamentos comerciais – GP3 para a coleta de RO, e Mindwave Mobile 2 para a coleta de dados de ativação neuronal. A ferramenta terá como output um dataset constituído de dados de EEG e RO coletados simultaneamente. O output será testado em um estudo de caso através da análise de performance de quatro diferentes algoritmos classificatórios treinados com o output para classificar entre duas possíveis atividades. 


\section{Objetivos}

Criar uma ferramenta capaz de gerar um dataset de EEG e ET coletado de forma síncrona e validar o dataset através da performance de algoritmos de aprendizado supervisionado treinados com ele.

\subsection{Objetivos Específicos}

\begin{enumerate}
    \item Criar código para coleta simultânea de EEG e RO;
    \item Criar dataset multimodal a partir da fusão de dados de EEG e RO coletados pela ferramenta e classificado de acordo com o estímulo apresentado ao participante;
    \item Treinar diferentes algoritmos supervisionados com o dataset multimodal gerado;
\end{enumerate}

\section{Justificativa}
Apesar da existência de técnicas que permitam a extração de mais de um modo de dados fisiológicos a partir de um equipamento apenas - como a extração da posição ocular a partir de assinaturas elétricas em dados de EEG, estes métodos necessitam de um grande volume de dados, o que exige equipamentos mais refinados e, por vezes, uma grande disponibilidade de tempo para criação dos datasets e deslocamento de participantes até a estação de coleta. A coleta de ET e EEG por equipamento comercial e acessível, seria, portanto, uma alternativa que permite um maior controle no desenvolvimento de estudos com algoritmos de aprendizado de máquina, sem depender de equipamentos de alto custo ou deslocamento de participantes até a estação. 


É argumentado que um maior acesso a construção de bases de dados multimodais poderá expandir e aprofundar os avanços em neurociência e estudos comportamentais, ao proporcionar um maior controle do design de experimento e expandir a quantidade de datasets fisiológicos gerados. 

\section{Organização do Documento}
O presente texto tem nove capítulos. O primeiro capítulo trata da contextualização do problema, objetivos gerais e específicos, e a justificativa para a abordagem selecionada. 

O segundo capítulo trata do referencial teórico, levantando pontos históricos importantes ao desenvolvimento desta pesquisa, uma introdução ao que seriam os sinais capturados pelos dois equipamentos de EEG e ET, características dos diferentes tipos de equipamento de captura e aborda a conversão de sinais analógicos para digital. 

O terceiro capítulo trata do processo da aquisição e do tratamento de sinais fisiológicos a serem classificados por algoritmos de aprendizado de máquina supervisionado – processos como remoção de ruído e seleção de características. Também aborda métodos de fusão de bases de dados de diferentes fontes - a nível de característica e a nível de decisão. 

O quarto capítulo apresenta os métodos de avaliação dos algoritmos classificatórios. 

O quinto capítulo aprofunda nos possíveis métodos de sincronização de coleta e de bases de dados.

O sexto capítulo introduz a proposta de ferramenta de coleta simultânea elaborada por código e aborda o método de sincronização off-line, com sincronização por código temporal. 

O sétimo capítulo aborda o cronograma.

O oitavo capítulo aborda os resultados esperados do estudo.


O nono capítulo apresenta as referencias do estudo.

\chapter{Eletricidade Cerebral}



\section{Correntes Elétricas}
%Gaspar, Alberto (2005). Física:Volume único. São Paulo: Editora Ática. 496 páginas. ISBN 9788508078837. Consultado em 12 de janeiro de 2012
A eletricidade é o nome dado a uma série de fenômenos que envolvem o fluxo de cargas elétricas (Gaspar, 2005). 
O potencial elétrico (também conhecido como tensão), é a quantidade de energia precisa para deslocar uma carga elétrica (Matias e Fratezzi, 2008). 
A diferença de potencial elétrico entre um material condutor gera um fluxo de 
cargas nomeado de \textbf{corrente elétrica} (Creder, 1989).
A intensidade do fluxo é medida em ampère (A) e determinada pela quantiadade de partículas que atravessam o seguimento
do condutor pelo tempo. As correntes podem ser contínuas ou alternadas, a depender se o sentido da corrente varia ou não; enquanto a corrente contínua 
é composta de polos, a alternada é composta de fases (Bhargava e Kulshreshtha, 1983). O calculo de corrente elétrica é 
dado pela seguinte equação:

\begin{equation}
    I = \frac{\Delta Q }{\Delta T},
\end{equation}

onde $\Delta Q$ é a quantidade de particulas que passam em um seguimento do condutor e $\Delta T$ indica o tempo. 

\section{Neurônios}
Os neurônios são células responsáveis pela condução de impulsos nervosos e se comportam como um “cabo eletrificado” - analogia
levantada a respeito da condução de impulsos elétricos, como apresentado no clássico estudo de Hodgkin e Huxley (1952), 
que teve como resultado uma modelagem dos potenciais de ação emitidos pelas células 
nervosas através de equações diferenciais não-lineares (figura 2.2). 
Os neurônios comunicam-se uns com os outros através das \textbf{sinapses}, onde ocorre a transmissão dos impulsos nervosos. 
É possível distinguir três partes anatomicas no neurônio: o corpo celular, o axônio e os dendrito (figura 2.1). 

\begin{figure}
    \centering
    \includegraphics[width=40mm]{corpo_celular.jpg}
    \caption{Anatomia de um neurônio com destaque para área de sinapse entre neurônios. Fonte: Bear (2015)}
\end{figure}

% Serway, R.A.; Jewett Jr., J.W (2008). Princípios de Física. 3. São Paulo: Cengage Learning. p. 909-910. ISBN 85-221-0414-X
A membrana celular permite a passagem de cargas elétricas através de canais, que podem ou não necessitar de energia para a movimentação
das cargas. \textbf{Capacitores} são dispositivos de polaridades diferentes nas extremidades, que armazenam
cargas elétricas num campo elétrico (Serway, 2008). Por sua capacidade de separar cargas elétricas entre o ambiente interno e externo, 
a membrana celular age como os capacitadores, com sua capacitância (habilidade de armazenar cargas elétricas), definida pela seguinte equação:

\begin{equation}
    C = \frac{Q}{\Delta V},
\end{equation}

onde $C$ é a capacitância, $Q$ é a quantidade de carga armazenada e $\Delta V$ é a tensão elétrica, medida em farad (F). 
\textbf{Resistência elétrica} diz respeito a capacidade de oposição a passagem de corrente elétrica e é medido em
 ohms ($\Omega$). Os canais da membrana podem se comportar como resistores, se opondo a passagem da corrente elétrica, e sua 
 resistencia pode variar dependendo das condições celulares, como por exemplo se o canal está abero ou  não. Na figura 2.2, $E$ representa
 \textbf{bateria} pois a concentração de ions (particulas elétricas) é diferente no meio intra e extracelular, graças ao trabalho de canais ativos (com custo de energia
 para manter esse diferencial). De forma simplificada, a corrente aplicada no neurônio pode injetar corrente no capacitor e também 
 ser distribuída pelos canais. Dado a definição de um capacitor, $I_c = d u /d t$, é possível definir a corrente elétrica em uma seção da 
 membrana como:


%https://neuronaldynamics.epfl.ch/online/Ch2.S2.html#Ch2.F2
\begin{figure}
    \centering
    \includegraphics[width=100mm]{modeloHH.jpg}
    \caption{Esquema Modelo Hodgkin-Huxley. 
    Dentro: indicando espaço intracelular; fora: indicando espaço extracelular (imagem a esquerda). Direita: 
    C = capacitor, R = resistor, E = Baterias. Na = Sódio, K = Potássio  Fonte: Neuromal Dynamcs (2014)} 
\end{figure}

\begin{equation}
    C \frac{d u }{ d t} =  - \sum_{k} I_k (t) + I (t),
\end{equation}

onde $u$ = voltagem ao longo da membrana e $t$ = tempo. 








\subsection{Impulsos Nervosos e Potencial de Ação}
Para passar informações, os neurônios geram \textbf{impulsos nervosos}, ou alterações no potencial elétrico de sua membrana. Este sinal elétrico
ocorre quando o estímulo recebido pelo neurônio ultrapassa um limiar de ativação, que desencadeia uma série de respostas celulares. A célula pode 
estar em repouso (com valor do interior celular em cerca de -70mV), passando por despolarização (quando ocorre um fluxo de cargas elétricas que faz com 
que o meio intracelular passe a ser positivo em relação ao meio extracelular), e em repolarização, quando a célula está retornando ao potencial de repouso,
como representado na figura 2.3. 

O aumento do inicial da voltagem é causado pela entrada de sódio através de canais dependentes de voltagem, que se segue 
pela perda de potássio e fechamento dos canais de sódio. 


\begin{figure}[!h]
    \centering
    \includegraphics[width=90mm]{bear_potencial.jpg}
    \caption[Impulsos nervosos conduzidos em neurônios]{Resumo do potencial de ação. Fonte: Bear (2015).}.\label{fig:potencial}
    \end{figure}

\clearpage

\section{Eletroencefalograma}
O conjunto de impulsos nervosos de grupos de neurônios geram campos magnéticos
 que podem ser captados por eletrodos colocados sobre a cabeça humana (Kandel, 2000). 
 Estes campos magnéticos foram primeiro registrados de coletas em humanos aproximadamente em 1929,
  em um experimento conduzido pelo psiquiatra alemão Hans Berger (İnce et al., 2021) – figura 2.2. 
  \begin{figure}[h]
    \centering
    \includegraphics[width=100mm]{serie_temporal_EEG}
    \caption[]{Primeiro EEG registrado em humanos, resultado do trabalho do psiquiatra Hans Berger. Fonte: Ince et al. (2021).} 
    \end{figure}


  Estes registros são o resultado dos potenciais de ação emitidos pelas células nervosas abaixo do eletrodo, 
  e permitem uma boa resolução temporal do comportamento nervoso (podendo atingir precisão de milissegundos), 
  mas em geral não permitem uma boa resolução espacial (como identificar a localização espacial do grupo celular responsável pela variação de voltagem observada). 



  

    \subsection{Tipos de Eletrodos para Captura de EEG}

    Existem diferentes tipos de eletrodos para a captura de EEG. Um resumo é apresentado na tabela 2.1. É notável também que com o desenvolvimento da capacidade computacional, novos recursos e métodos para a análise destes dados vem sendo benéficos à construção do conhecimento científico, agora também contando com o desenvolvimento de algoritmos de aprendizado de máquina, aprendizado profundo e inteligência artificial. 
  
\subsection{Sistema Internacional 10/20 de Posicionamento de Eletrodos}

A técnica de registro de EEG vem sendo desde então aperfeiçoada e escolhida em investigações 
comportamentais devido a sua natureza não invasiva.
 Um exemplo de aperfeiçoamento foi a criação de um sistema internacional de posicionamentos 
 de eletrodos para a coleta de EEG – o sistema 10/20 (Klem et al., 1999), 
 representado na figura 2.3. 
 O registro capturado nos eletrodos advém de uma 
 diferença de potencial elétrico. Esta diferença pode ser em 
 referência à um eletrodo colocado em uma região externa ao escalpo (como orelha), 
 ou à uma voltagem média comum (Tavares, 2011).




%\subsection{Potenciais Relacionados a Eventos}




\chapter{Rastreamento Ocular}

\section{Anatomia Ocular}
O globo ocular é majoritariamente opaco, com exceção da córnea, que é transparente. 
A pupila é a região que da passagem para a luz e possui diâmetro variável. Os músculos da íris são os que controlam a dilatação da pupila. 
A focalização da imagem deve se concentrar na fóvea, onde se encontram células muito sensíveis a luz (Helene e Helene, 2011). 
A fixação ocular compreende a um período de cerca de 100 milissegundos onde o olhar se fixa em um ponto de convergência (Barreto et al., 2012). 
Este período se encerra com o movimento de sacada, que compreende ao movimento rápido até uma nova fixação do olhar em outro local.
Através da coleta do posicionamento ocular, é possível calcular uma taxa de dispersão focal ao longo do tempo e piscadas. 
Estes dados foram previamente correlacionados com estados emocionais (Soleymani et al., 2012) e 
também aplicados em estudos com algoritmos de aprendizado de máquina e deep learning. Barreto (2012)
resumiu alguns dos principais termos utilizados em pesquisas de rastreamento ocular (RO):

\begin{figure}[h]
    \centering
    \includegraphics[width=70mm]{anatomy.jpeg}
    \caption[]{}\label{fig:}
    \end{figure}

\section{Equipamentos de Rastreamento}
Para detectar onde o participante está focando seu olhar ao longo do tempo, alguns equipamentos de ET
 fazem uso de luz infravermelha e câmeras de alta definição que projetam a luz
  diretamente no olho do participante e gravam a direção do olhar a partir do reflexo. 
  Como a luz infravermelha abrange um comprimento de onda não detectável pelo olho humano, 
  o direcionamento desta luz no olho não interfere visão do participante. 
  O cálculo do direcionamento ocular é feito com base em algoritmos próprios de cada fabricante. 
  Existem alguns tipos de equipamentos de rastreamento ocular. São eles: (1) Webcam, (2) Vestível (Werable) e (3) Baseados em Tela. 
  Webcam diz respeito a equipamentos não especializados para o uso de rastreamento; usáveis correspondem a equipamentos como óculos de rastreamento ocular 
  e realidade virtual, e os baseados em tela dizem respeito aos equipamentos de coleta especializada que podem ser acoplados a um computador Tobii Pro (2020).

  \section{Eletrooculograma} 
Fonte: https://eyewiki.aao.org/Electrooculogram

  Definition
  The electroocoulogram (EOG) is an elecrophysiologic test that measures the existing resting electrical potential between the cornea and Bruch's membrane. The mean transepithelial voltage of bovine Retinal pigment epithelium is 6 millivolts (mV). [1]
  
  History
  The EOG was described and named by Elwin Marg in 1951. Clinical applications were described first by Geoffrey Arden in 1962, who realized that the most valuable information was the comparison of the amplitudes under light and dark-adapted states (the Arden ratio).
  
  Testing process
  The patient should be dilated. The amount of light passing through the pupils is measured in a unit called Trolands (the product of luminance (cd/m2) and pupil area (mm2)).[2] Thus the pupillary diameter may change the needed luminance for the same effect on the retina.
  
  The patient should be in stable indoor lighting for at least 30 min before the test. Strong retinal illumination including retinal imaging (fluorescein angiogram, fundus photography and others) and indirect ophthalmoscopy should be avoided during this period.
  
  The patient is told to remain still other than moving his/her eyes back and forth. Four recording skin electrodes (silver-silver chloride or gold-disk) are placed at the medial and lateral canthi of both eyes, and the grounding electrode is placed on the forehead. 
  
  Principle of EOG
  The difference of electrical potential of the anterior and posterior part of the eyeball is called the standing potential.[2] Standing potential indirectly measures the transepithelial potential (TEP) of the retinal pigment epithelium (RPE). TEP is the difference of membrane potential of basolateral and apical membranes of RPE.
  
  The standing potential may be determined in 2 ways:
  
  EOG- determines function of outer retina and RPE. It has positive waveform in light and negative waveform in dark. It is recorded during 15 min of dark adaptation and 15 min of light adaptation. During 15 min of dark adaptation the standing potential usually reaches a minimum level (dark trough/DT) at 10-15 min. During 15 min of light adaptation the standing potential achieves the highest value at 7-12 min called a light peak/LP. The LP results from increased free intracellular calcium released from the endoplasmic reticulum. The role of bestrophin of endoplasmic reticulum and L type calcium channel of basolateral membrane is crucial in this. The increased intracellular calcium opens the 'Calcium dependent light peak chloride channels' of the basolateral membrane through which negative chloride ions are extruded from the RPE and the RPE depolarizes in light.[3]
  Fast Oscillations (FO)- Is an optional additional test performed using alternate 1 min dark and light periods. It has negative waveform in light and positive waveform in dark. This opposite waveform compared to EOG is related to different mechanism of FO and shorter dark and light periods. At the onset of light, there is a decrease in potassium levels at the subretinal space. This creates a strong outward hyperpolarising potassium current from the apical membrane of RPE which results in the c wave of electroretinogram. The chloride channels at the basolateral membrane (CFTR) of RPE may have important role in the generation of the FO, which may be reduced in cystic fibrosis.[4] The wave form of FO is sinusoidal compared to the EOG which has a shape of plateau after post hoc DC restoration by digital integration,
  Components of the EOG
  The light-insensitive component accounts for the dark trough and is dependent on the integrity of the retinal pigment epithelium (RPE) as well as the cornea, lens, and ciliary body. The light-sensitive component is the slow light rise of the EOG and is generated by the depolarization of the basal membrane of the RPE.
  
  Reporting of EOG
  According to the 2017 ISCEV standards,[2] the report of EOG should include
  
  Light peak: dark trough ratio (this terminology is preferred over conventional Arden ratio)
  Amplitude of dark trough (mv)
  Time from the start of light phase to light peak (when present)
  type of adapting light source
  pupil size
  Difficulties/deviation from protocol including patient compliance, inconsistent eye movements
  Interpretation of Results
  The Arden ratio, the ratio of the Light peak (Lp) to dark trough (Dt) is used to determine the normalcy of the results.
  
  An Arden ratio of 1.80 or greater is normal, 1.65 to 1.80 is subnormal, and < 1.65 is significantly subnormal.
  
  
  \subsection{Artefatos em EEG por Movimentação Ocular}

  %https://www.ncbi.nlm.nih.gov/books/NBK390358/
  \begin{figure}[h]
    \centering
    \includegraphics[width=130mm]{artefatos_oculares_EEG.jpeg}
    \caption[]{Artefatos oculares em registro de EEG. Fonte: Britton et al. (2016)}\label{fig:}
    \end{figure}



\chapter{Séries Temporais}

Séries temporais são sequencias de pontos ao longo do tempo, onde pontos vizinhos são dependentes (Ehlers, 2007). 
As séries podem ser contínuas ou discretas. Um exemplo de sinal contínuo é a diferença da voltagem de neurônios
capturada por eletrodos ao longo do tempo. 

\chapter{Sincronização}

O uso de equipamentos com função exclusiva de sincronização para coletas simultâneas é comum em pesquisas ambientes academicos e clínicos.
A proposta de oferecer maior acessibilidade através da redução de custo e desenvolvimento de novas tecnologias encontra, portanto, um desafio a respeito 
de como realizar a sincronização dos dados fisiológicos sem abrir mão da praticidade e custo dos equiapementos desenvolvidos. 
Algumas propostas já foram exploradas a respeito, como o uso de piscadas e código temporal para garantir a sincronização de EEG e ET (Bækgaard et al. 2015, Notaro et al. 2018).

\section{Frequência de Coleta}
Como os sinais análogos são convertidos para sinais digitais, existe uma perda de informação por esta conversão. 
A \textbf{resolução de frequência} mede o espaço entre duas frequências. 

$$srate/N$$

Srate = sampling rate 
N = Número de amostras

\subsection{Frequência Nyquist}
É a frequência mais rápida onde o sinal pode ser medido, onde é estabelecido que a maior frequência que podemos medir é a metade 
da frequência de coleta.


\section{Sincronização com Timecode}
Notaro et al. (2018) faz uso do código temporal, ou \textit{timecode}, para sincronizar dados de EEG, ET e dados comportamentais 
coletados de participantes enquanto estes faziam atividades de um site de aprendizagem de linguas. O driver
do fabricante do equipamento comercial de EEG utilizado permite alteração da latência da coleta de dados, que
foi modificada do valor padrão de 16 milissegundos para 1 millisegundo, afim de aumentar a precisão do equipamento.
A informação da ocorrência de clicks no site foi retina na forma de milissegundos (HH:MM:SS:MsMsMs), e esta informação foi utilizada 
para sincronizar dados de ET, EEG e movimentação de mouse. 

\section{Sincronização com Piscadas}
Piscadas duram cerca de 200 milissegundos em média e podem indicar estados de alerta (Caffier, 2013). Piscadas também aparecem 
em dados de EEG de forma característica, podendo alcançar uma amplitude de sinal acima de 200 microvolts em eletrodos próximos a órbita ocular (Hoffmann e Falkenstein, 2008). Assim sendo, é possível realizar uma sincronização por piscadas ao se detectar 
o movimento em ambos os equiapmentos de coleta. No caso do EEG, as piscadas são comumente descartadas como artefatos indesejáveis. Já no estudo de 
Bækgaard et al. (2015), elas são a assinatura de sincronização entre os equipamentos de coleta de EEG e ET em função de sua onda característica (geralmente muitos milivolts acima do sinal do EEG), e de também 
ser detectdo através dos equipamentos de rastreamento ocular.

\subsection{Identificação no Sinal do EEG}
Para se detectar a piscada através de um sinal, é possível tentar realizar o método de \textit{Independent Component Analysis}, ou análise de 
componente independente, mas as características do sinal de piscada também permite outras abordagens, como a identificação por função de probabilidade.
Considerando o movimento de maior característica da piscada, é preferível se calcular a probabilidade do movimento de fechar os olhos 
ao movimento de abertura, em função de uma variação em tempo ser mais comumente encontrada na fase de abertura (Caffier, 2013).

\subsection{Identificação no Sinal de ET}
Como o equipamento de rastreamento procura encontrar sinais da movimentação ocular, ele também detecta a ausencia desse sinal. No estudo 
de Bækgaard et al. (2015), uma perda de até 500 milissegundos foi considerada como indicador da ocorrência de uma piscada. 

\subsection{Correlação Cruzada}


\section{Códigos para Sincronização}
Alguns equipamentos podem se beneficiar da existencia de \textit{toolboxes} ou bibliotecas direcionadas à sincronização. É o caso 
dos equipamentos Tobbii na solução de EEG-Eye para a linguagem MATLAB. Uma forma de se fazer sincronização é atrav´s 


\chapter{Acesso a Datasets Fisiológicos}

Apesar da existência de fontes de datasets fisiológicos, 
o número de datasets disponíveis ainda é restrito. Sobre a qualidade dos datasets, 
Mendoza et al. (2021) observou que nenhum dos nove datasets disponíveis publicamente para 
treinamento de algoritmos de classificação de emoção analisados possuía todos os critérios de 
referência levantados por estudos anteriores. Apesar da ausência dessas referencias, os datasets 
apresentaram uma base para desenvolvimentos futuros. Sobre a disponibilidade de datasets fisiológicos,
Rim et al. (2020) faz uma análise de datasets públicos e privados.
 No exemplo apresentado na figura 3.1, 
 é possível observar que datasets públicos combinando sinais são minoria nas diferentes fontes de dados analisadas. 

 O campo de neurociência computacional é um dos diversos campos beneficiados 
 com o desenvolvimento da tecnologia, que constantemente melhora no sentido 
 de propor novas ferramentas de captura de sinais fisiológicos e novas formas de
  processá-los. Diferentes equipamentos de EEG e ET implicam em uma diferente forma
   de se montar a coleta e processar os dados coletados.
  Embora surjam novos métodos, importantes considerações devem ser feitas a 
  respeito da resolução de captura, de forma a não se deixar perder informação desejada.   

\section{Aquisição de Dados Fisiológicos}

Um exemplo de como se realizar a montagem para coleta de EEG e ET é demonstrado na figura 3.1, utilizado na montagem do dataset EEGEyeNet (Kastrati et al., 2021),
onde o participante é colocado de frente para o monitor para apresentação de estímulos com o equi
pamento de coleta de EEG sobre a cabeça e o aparelho de ET direcionado aos olhos do participante. A piscada é comumente
removida como artefato indesejável nos dados de EEG, fazendo parte de muitos pré-processamentos de estudos com EEG e ET (Hosseini, 2020).
Entretanto, ET e EEG podem ser sincronizados a partir da assinatura da piscada, permitindo uma correção contínua dos dados (Bækgaard e Larsen, 2014). 
Outras formas de sincronização de EEG e ET também foram propostas, como por código temporal ou com auxílio de equipamentos externos – exploradas adiante. 
 
Figura 3.2. Setup de coleta de EEG e ET. Fonte: Kastrati et al. (2021)
A fusão de dados pode ser realizada no nível de característica ou feature,
assim como a nível de decisão do algoritmo classificatório (Klein, 2014; Mendes et al., 2016).  explicadas adiante.
      Na figura 3.2. é possível observar um fluxo de processamento dos sinais capturados por diferentes sensores.  Os diferentes 
      sensores representados no estudo de Mendes et al. (2016) são aqui representados pela coleta de EEG e ET.
 

Figura 3.2. Exemplo de Fluxo para Fusão de Dados de Sensores. Fonte:  Mendes et al. (2016).
Na Feature Level Fusion (FLF), os dados de diversas fontes são extraídos dos sensores e unidos de forma a 
gerar um vetor único com informações multimodais; no Decision Fusion (DL) a classificação ocorre para cada categoria 
de fonte de dado (exemplo: uma classificação para EEG e outra para ET) e estas classificações são combinadas em um esquema de voto (exemplo: a classificação mais comum) para se chegar em uma categoria final (Bota et al., 2020). A respeito de qual formato seria melhor, Bota et al. (2020) explorou o assunto para cinco bases de dados fisiológicos classificados de acordo com o estímulo emocional apresentado ao participante e observou que o melhor método de fusão é altamente correlacionado à base dados, embora o FLF tenha sido escolhido como o melhor em função de sua baixa complexidade em relação ao DF. 
No estudo de Kastrati et al. (2021), dados de EEG e ET foram coletados por equipamentos com 500Hz de resolução e si
ncronizados por código, com auxílio do Eye EEG Toolbox para MATLAB. A sincronização foi confirmada pelo início de s
inais registrados em ambos os equipamentos, apresentando erros menores que dois milissegundos. 


\chapter{Aplicação de Datasets Fisiológicos}

O pré-processamento de EEG consiste em remoção de artefatos, 
tais como contração muscular e movimentação ocular.
Um exemplo de pipeline de processamento para dados de EEG é apresentado na figura 3.3.
A etapa de extração de características consiste em, partindo dos dados com 
remoção de artefatos indesejáveis, extrair métricas estatísticas, como média, 
mediana e desvio padrão aplicados a uma janela de tempo, ou outras métricas, como entropia de Shannon 
(como feito no estudo de Thapaliya et al. (2019)). A seleção de features pode 
envolver o uso de algoritmos que permitem reduzir o número de características a 
serem apresentadas como input ao algoritmo, como o Principal Componen Analysis (PCA).
 A partir dessas etapas, os dados seguem são comumente divididos entre treinamento e 
 teste, para validar o algoritmo ou algoritmos a serem estudados. 
 O estudo de King et al. (2017) apresenta alguns exemplos de informações que podem ser extraídas de sinais fisiológicos capturados por sensores no quadro 3.1.s

\section{Avaliação de Algortimos}

Após entender como deverá ser realizado o pré-processamento dos dados fisiológicos multimodais e sua união, 
é necessário entender de qual forma a avaliação do melhor método de sincronização será realizada. 
No presente estudo, o objetivo esperado é de se encontrar o melhor
método de integrar os dados multimodais como sendo aquele que obtêm uma maior acurácia dentre os 
algoritmos selecionados. O presente capítulo introduz o conceito de classificadores lineares,
não lineares e das métricas de avaliação de algoritmos classificatórios.


Algoritmos classificatórios
podem ser lineares ou não lineares. 
Classificadores lineares conseguem separar 
as categorias de dados em uma reta no espaço vetorial, seja ela com uma ou mais dimensões (reta, plano ou hiperplano). Um exemplo do que seriam dados linearmente separáveis e não podem ser observados na figura 4.1.
Alguns algoritmos classificatórios bastante utilizados são introduzidos adiante. Para problemas mais complexos,
é comum o uso de algoritmos que classifiquem dados não lineares, como o Support Vsector Machine (SVM), K-Nearest Neightbor (KNN), 
Rede Neural Artificial (ANN) e Regressão Logística (RL).

Para a avaliação do algoritmo, a acurácia e precisão oferecem métricas para avaliar o erro observado do output (ou resultado) do modelo. 
Para isso, é necessário saber o valor real e o valor estimado pelo algoritmo classificatório. 
A acurácia mede a proximidade de um determinado valor e o valor de referência (ou valor real) (equação 4.1). 
A precisão mede a dispersão dos valores obtidos pelo modelo (equação 4.2). Um bom algoritmo é preciso e possui alta acurácia. 

A performance de algoritmos classificatórios pode ser 
melhor observada através de uma matriz de confusão. 
Esta matriz permite observar onde o algoritmo mais erra, se em 
classificar verdadeiros positivos ou verdadeiros negativos. 
A matriz do exemplo é utilizada para classificadores binários,
embora uma versão desta matriz possa ser utilizada para classificadores multicategóricos. 


Em seu estudo sobre o uso de algoritmos para classificação de emoções a partir de dados fisiológicos, Zheng et al. (2014) coletou dados de dilatação da pupila, movimentação ocular e EEG para identificar qual seria a classificação do estímulo emocional apresentado aos participantes. O processo de coleta do estudo pode ser observado na figura 2. A classificação do estímulo apresentado (vídeo clips de 4 minutos de duração) obteve acurácia máxima de 73.59% de dados coletados em 12 sessões de experimento, onde, em cada sessão, os 5 participantes assistiram a 15 vídeos (5 de emoção neutra, 5 de positiva e 5 de negativa). 
 
Figura 5.4. Design de Experimento para Coleta de EEG e ET. Fonte: Zheng et al. (2014).

Lu et al. (2015) também faz uso de dados de EEG e RO para classificação de emoções nas três valências emocionais eleitas no estudo de Zheng et al. (2014). Em contraste com o volume de informações coletadas no estudo de Zheng et al., Lu et al. coletam uma maior quantidade de dados de rastreamento ocular – extraindo 16 métricas de RO, enquanto o estudo de Zheng foca em apenas métricas principais da dilatação ocular. Os resultados da acurácia do algoritmo aplicado aos diferentes métodos de fusão de dados multimodais estão resumidos na imagem 5.5, ficando evidente que, independente do método utilizado para fusão das modalidades de EEG e RO, as melhores acurácias foram encontradas para base de dados de mais de uma fonte de informação fisiológica. 
 
Figura 5.5. Acurácia por Método de Fusão de Modalidade e Modalidade Única em Algoritmo Supervisionado. Fonte: Lu et. al. (2015).
No trabalho de Thapaliya et al. (2019) dados de EEG e ET foram aplicados em algoritmos de máquina, com o objetivo de estudar uma melhora no método de diagnóstico de crianças com autismo através de diferentes formas de pré-processamento (exemplo de processamento do estudo na figura 5.6). Os dados de EEG tiveram suas métricas estatísticas coletadas para a construção de um vetor de características (incluindo desvio padrão e média por janela de tempo dos dados de EEG filtrados), assim como a entropia calculada por janela temporal. Para os dados de RO, os tempos de fixação foram coletados, em conjunto com o resultado de testes cognitivos. 
 
Figura 5.6. União de dados de EEG e ET. Fonte: Thapaliya et al. (2019).
Em seu estudo, diferentes métodos de construção de vetores de características foram analisados, tanto para os dados unimodais quanto para a junção de EEG e ET. Através das acurácias apresentadas para os diferentes métodos de processamento, podemos observar que determinados algoritmos aumentaram sua acurácia a depender do modo no qual o vetor de características foi construído. Por exemplo, enquanto o algoritmo Support Vector Machine (SVM) atingiu 71% de acurácia com o vetor que incluiu Entropia para as janelas de EEG e uso do Principal Component Analysis (PCA), a regressão logística com maior acurácia foi atingida com o dataset de desvio padrão de EEG e dados de rastreamento ocular sem a aplicação de PCA (Thapaliya et al., 2019). 
Lim e Chia (2015), estudaram a correlação de ondas EEG detectadas em um equipamento de eletrodo único e estresse cognitivo induzido pelo teste de Stroop. A análise foi feita com base na aplicação de três algoritmos: Artificial Neural Network, k-Nearest Neighboor (KNN) e Linear Discriminant Analysis (LDA), dos dados de EEG transformados pela aplicação da Transformação Cosseno Discreta (Discrete Cosine Transform – DCT). O KNN com o DCT conseguiu classificar melhor o estado de estresse do participante.
O uso do MindWave Mobile 2 foi recentemente empregado para o controle de cadeira de rodas (Abuzaher e Al-Azzeh, 2021; Permana et al., 2019), controle de mão robótica e robô móvel (Purnamasari et al., 2019; Rusanu et al., 2019; Rușanu et al., 2021) e predição de personalidade (Bhardwaj et al., 2021). Outro estudo com uso de eletrodo único como fonte de dados eletrofisiológicos foi o trabalho de Quesada-Tabares et al. (2017), onde foi demonstrado que o uso de EEG comercial e com eletrodo único também possui um importante poder classificatório quando aplicado em algoritmos. Em seu estudo, sete participantes observaram imagens selecionadas do International Affective Picture System (IAPS) pertencentes a três grupos com diferentes valores de valência e excitabilidade. O teste de ANOVA aplicado indicou uma diferença estatisticamente significante entre os sets de imagens. Um segundo teste foi conduzido pela aplicação de um algoritmo de classificação no estilo árvore de decisão, chegando a uma acurácia média entre os sete participantes de 80.71%. Os estudos foram analisados pelo MATLAB.
Bos (2021) também explora o uso do MindWave no contexto escolar para medir a atenção dos alunos. Em seu estudo, o nível de atenção com alunos assistindo a um vídeo educacional sem e outro com interações (fazendo pergunta aos alunos) é explorado e a distribuição percentual das diferentes bandas de frequência são comparadas entre os grupos. Bos (2021) observou uma relativa diminuição banda de frequência de onda beta para o grupo que não assistiu ao vídeo interativo, o que foi relacionado a um processamento cognitivo reduzido e menor atenção.
Bhardwaj et al. (2021) analisou o uso dos dados coletados com o MindWave para classificar sete traços de personalidade com o algoritmo deep long short term memory (DeepLSTM) e tratando os dados com transformada de Fourier Rápida. A pesquisa contou com 50 participantes (25 mulheres e 25 homens), com idades entre 18 e 46 anos, ao longo de cinco dias, e os dados foram coletados enquanto os participantes assistiam vídeos relacionados a traços de personalidade. Os traços foram separados de acordo com os tipos de personalidade definidos no Myers-Briggs Type Indicator, e ao final de cada vídeo, o participante deveria dizer se concordavam, discordavam ou eram neutros aos questionários de personalidade sobre o traço proeminente no estímulo. O questionário de cada participante foi utilizado para determinar o traço de personalidade, que serviria para então classificar os dados em três possíveis outputs: (a) participante tem traço de personalidade apresentado no vídeo, (b) participante não tem traço apresentado de forma significativa e (c) participante tem traço oposto ao apresentado no vídeo. 
5.6 CONSIDERAÇÕES FINAIS
Datasets multimodais tendem a performar melhor em algoritmos classificatórios que datasets unimodais. 
A forma de processamento dos dados também pode ter impactos na performance classificatórias dos algoritmos. 
Equipamentos comerciais já foram previamente utilizados em estudos de algoritmos classificatórios. 
É esperado que um método de fusão eficiente reflita em uma maior acurácia dos algoritmos treinados no dataset. 
Para comparar a eficácia de um determinado método de construção de bases de dados, 
cada uma das bases geradas no presente projeto terá a acurácia calculada e comparada com as demais bases de dados.





Outra forma de avaliação é através da curva ROC, 
ou Característica de Operação do Receptor. A curva é obtida a
o se observar a variação da taxa de verdadeiros positivos (sensibilidade, ou Positivos Verdadeiros / Positivos Totais) em função de 1 – 
especificidade, ou taxa de falsos positivos (Positivos Falsos / Negativos Totais). 

\chapter{Materiais e Métodos}

\section{Construção da Ferramenta}


Dois equipamentos comerciais foram utilizados para construção da ferramenta de coleta sincronizada: Mindwave Mobile II e o
GP3; para a coleta de EEG e ET, respectivamente. O código de coleta que gera o dataset
multimodal foi construído em MATLAB, e o pré-processamento dos dados antes de serem \textit{inputs} no treinamento
de algoritmos no software Orange foi construído em Python. O presente capítulo irá apresentar os equipamentos de coleta
e o método adotado para construção da ferramenta. 

\subsection{Mindwave Mobile II}
O equipamento possui um eletrodo de coleta e um eletrodo de referência que ficam posicionados 
acima da sobrancelha esquerda e na orelha esquerda do participante, respectivamente. 
A posição do eletrodo de coleta em relação ao sistema de referência de posição de eletrodos (10-20), é o FP1, 
correspondendo a região Frontopolar 1. A coleta de dados do aparelho se dá por conexão via \textit{bluetooth} e 
funciona em computadores Mac, Windows ou celulares Androids ou iOS, disponíveis em um raio de 10 metros (informações do fabricante). 
Ele coleta ondas cerebrais variando entre 3 e 100Hz, com uma frequência de 512Hz (NeuroSky Inc., 2015). 

O aparelho automaticamente distingue os dados coletados em ondas alfa, beta, gama, teta e delta; 
além de coletar informações subjetivas no formato de medidas de atenção e meditação, 
por meio de um algoritmo de reforço de aprendizado não disponibilizado ao publico. 
Também mede a ativação muscular próxima ao eletrodo para estimar a qualidade do sinal. 
O MindWave Mobile filtra interferência elétrica e converte o sinal detectado pelo eletrodo em sinal digital. 
O chip que faz o filtro e conversão se chama ThinkGear, e permite a filtragem de ruído para interferência ativação muscular (EMG) e 50/60Hz de corrente alternada. 

\subsection{Gazepoint GP3}

O GP3 é um equipamento comercial de coleta do movimento dos olhos, 
fabricado pela Gazepoint. Possui software próprio para análise dos dados, 
além de ser possível realizar coleta de dados com linguagens de programação open-source. O GP3 funciona emitindo uma luz infravermelha (IR) 
diretamente nos olhos do participante e captando a reflexão da luz para localizar o ponto focal ao longo do tempo. 
Permite coletar a direção do olhar, número de fixações, tempo até a primeira fixação, taxa de piscadas,
 duração de piscadas, diâmetro da pupila, tempo de duração do olhar em um determinado ponto focal, 
 objetos observados em uma imagem, entre outros (informações do fabricante).

O Gazepoint GP3 estabelece sua conexão com o computador através de dois cabos USB - um cabo de energia e outro para dados.
Seu posicionamento ideal é logo abaixo do monitorde estímulo. Para um melhor posicionamento, o fabricante 
sugere uma distância ideal de 65 cm dos olhos do participante até o equipamento. O GP3 possui as seguintes características, conforme
especificado pelo fabricante:

\begin{itemize}
    \item Acurácia de 0.5-1 grau de ângulo visual
    \item 60 Hz de frequencia de atualização
    \item calibração de 5 e 9 pontos
    \item API
    \item Captura movimento de 25cm horizontais e 11cm verticais
    \item 15 cm de limite de profundidade de movimento
\end{itemize}

Para poder realizar a coleta dos dados, é necessário manter o Gazepoint Control (API do desenvolvedor) ligado. 

\subsubsection{Calibração GP3}
Uma calibração é realizada pela própria API do equipamento, afim de estabelecer qual o apontamento ocular do participante. 
A calibração pode ser feita em 5 pontos ou 9 pontos no monitor de exibição de estímulo. Os pontos na tela são apresentados
em sequencia e o participante deve acompanha-los com o olhar até a finalização da calibração. 

Após a calibração ser concluída, a API calcula o erro do sistema em relação ao olhar para o olho esquerdo (em verde) 
e direito (vermelho). 

\subsubsection{Dados Capturados pelo GP3}

\textbf{Fixação:} É um agrupamento de pontos focais do olhar que duram entre 20-300 ms (Brand, 2020).

\textbf{Gaze Point:} Gaze point é o ponto focal do usuário em um dado momento. No equipamento GP3 é gravado um ponto focal a cada aproximadamente 17 milisegundos.
O ponto de gaze é gravado em relação as coordenadas x e y, que servem para identificar a posição do olhar na tela de experimento. 

%Anjith George and Aurobinda Routray, “A score level fusion method for eye movement biometrics,” Pattern Recognition Letters, vol. 82, pp. 207–215, 2016
\textbf{Sacada:} Compreende a um movimento rápido dos olhos após a fixação, e pode ser medida através de pixels por segundo.
O valor limite entre sacada e fixação é a velocidade de 1.8 pixels por segundo (George e Routray, 2016), onde acima
é uma sacada e abaixo, uma fixação.




\section{Coleta}
Para a coleta simultânea, duas linguagens de programação foram utilizadas: MATLAB e Python. 
A montagem do \textit{setup} foi feita com base nas coletas realizadas para a construção dos datasets EEGEyeNet e 
 ZuCO. 

Inicialmente, o software Gazepoint Control foi ligado para rodar em \textit{background} no computador. 
Em seguida, o MindWave mobile II foi ligado e teve sua conexão \textit{bluetooth} estabelecida e verificada.
A porta de conexão (porta COM) PC-Mindwave pode mudar, então foi verificado em qual porta foi estabelecida a conexão 
em todo início de nova coleta. Após a verificação da porta, foi observado se houve necessidade ou não de alterar
este valor na variável do código. 
    
 
 
 O Gazepoint
  ficou distante do participante até o software próprio acusar distância ideal
   (sinalizado por um sinal verde no topo do software de regulação do equipamento). 
   Para teste de coleta foram utilizados dois monitores: um para calibração e apresentação de imagens, 
   e outro para o desenvolvimento de código e testagem (figura 8.1) 


\begin{figure}
    \centering
    \includegraphics[width=130mm]{Screen Shot 2022-08-14 at 00.54.17.png}
    \caption{Montagem de Estudo de Caso da Ferramenta de Coleta EEG-ET.}
\end{figure}


%Tabela 6.1 Caracteristicas Montior utilizado na Coleta de Eye Tracking 
%Monitor	Video Interno conectado a Intel® HD Graphics 630
%Resolução da área de trabalho	1920  x 1080
%Resolução do sinal ativo	1920  x 1080
%Taxa de atualização	60 Hz
%Intensidade de bits	6 bits
%Formato de cor 	RGB
%Espaço de cores	Alcance dinâmico padrão

\section{Criação do Dataset}
O equipamento GP3 possui uma \textit{Application Programming Interface} (API) própria, 
o Open Gaze API. 
A API é uma alternativa de se controlar o equipamento sem precisar do software 
pago tambem feito pela Gazepoint. A API utiliza de um \textit{Transmission Control Protocol – Internet Protocol (TCP/IP) 
socket}, que permite a comunicação entre a aplicação e o servidor (fonte dos dados de ET). 
O IP determina o endereço para o qual os dados serão enviados, e o TCP utiliza a arquitetura de 
rede para realizar o transporte. O formato de dado utilizado para a API é o \textit{Extensible Markup Language} (XML), 
que também pode ser implementado e estabelecer conexão com Python. 
As portas utilizadas para a comunicação de forma padrão são: 
localhost (127.0.0.1) e port 4242. 

Na API do Gazepoint, o cliente tem duas tags de comunicação: GET e SET. 
Ao utilizar o SET o cliente pode alterar o valor de alguma variável. 
O comando GET não tem a possibilidade de alterção de valores. O servidor 
pode enviar dados para o cliente com diferentes tags: ACK, NACK, CAL e REC. 
As duas primeiras são geradas em resposta aos comandos de GET e SET (ACK – Sucesso e NACK –Falha). 
Os CAL são gerados com bases nas calibrações e REC serve para os dados gravados. 
Estas regras de escrita são utilizadas pelo codigo em Python para estabelecer o controle do GP3,
 permitindo que o código calibre o equipamento e estabeleça definições de variáveis.

Para o funcionamento adequado do Mindwave, é necessário instalar uma tecnologia chamada \textit{ThinkGear} (TG), 
que permite a troca de informações entre o equipamento e os softwares compativeis e processa o 
sinal detectado pelo eletrodo. O TG também é responsável pelo cálculo dos chamados \textbf{eSense Meteres}, 
correspondendo aos dados de Atenção e Meditação em uma escalda de 0 a 100 (Neurosky, 2018). 



\section{Dicas para o capítulo}
Dicas importantes que devem ser contempladas neste capítulo, segundo~\cite{marconi.lakatos:2003}:
\begin{itemize}
\item Verificar se o capítulo responde as seguintes questões: Como? Com quê? Onde? Quanto?
\item A linguagem do projeto deve ser escrita com tempo verbal no futuro e da dissertação no passado.
\item É importante mencionar sobre: tipo de pesquisa (bibliográfica, descritiva, documental, experimental etc), dados (fonte de dados, forma de obtenção), população e amostra, tratamento e análise dos dados (descrição mais detalhada do método -- ou métodos -- que serão utilizados), limitações da pesquisa.
\end{itemize}

\section{Observações Sobre Quadros e Tabelas}

Quadros e tabelas são de uso semelhante às figuras, no que diz respeito à numeração, uso de legenda, e necessidade de citar ao menos uma vez antes da ocorrência. No entanto, no caso dos quadros e tabelas a legenda deve ser colocada acima, e não abaixo como nas figuras.

A Tabela~\ref{tab:exemplo} ilustra esse uso. Observe que a citação de uma tabela específica (pelo número) é com a palavra ``tabela'' em maiúscula, ao contrário da referência a tabelas em geral. Note que em uma tabela as bordas são horizontais (não use bordas verticais para separar colunas), e não são necessárias bordas para separar cada linha. Separe apenas as linhas do início, fim, e dos indicadores dos campos presentes, como no exemplo. Podem ser usadas bordas horizontais para separar regiões distintas de dados (seções de dados), se necessário.

\begin{table}[h]
\centering
\caption{Parâmetros utilizados na implementação do método de deteção de bordas proposto, em cada configuração considerada.}\label{tab:exemplo}
\begin{tabular}{ccllll}
\cline{1-5}
\multirow{2}{*}{Configuração} && \multicolumn{3}{l}{\hspace*{12pt}Parâmetro}&  \\
&& \hspace{4pt}A\hspace{4pt} & \hspace{4pt}B\hspace{4pt} & \hspace{4pt}C\hspace{4pt} & \\ \cline{1-5}
1                             && \hspace{4pt}10\hspace{4pt}        & \hspace{4pt}5\hspace{4pt}       & \hspace{4pt}2\hspace{4pt}       &  \\
2                             && \hspace{4pt}20\hspace{4pt}        & \hspace{4pt}5\hspace{4pt}       & \hspace{4pt}3\hspace{4pt}       &  \\
3                             && \hspace{4pt}30\hspace{4pt}        & \hspace{4pt}8\hspace{4pt}       & \hspace{4pt}5\hspace{4pt}       &  \\ \cline{1-5}
\end{tabular}
\end{table}

O Quadro~\ref{quadro:exemplo} é um outro exemplo. Note que um quadro se diferencia de uma tabela pelo uso de campos fechados, por meio de linhas horizontais e verticais. As tabelas são mais usadas para dados quantitativos, enquanto quadrados são mais usados quando há descrições textuais (mesmo que haja dados quantitativos também).

\begin{quadro}
\caption{Exemplo de um quadro (retirado de~\cite{Gomes2011}): \emph{Variáveis explicativas que representam características socioeconômicas dos idosos.} Fonte:~\cite{Gomes2011}}\label{quadro:exemplo}
\begin{center}
\scalefont{0.705}
\begin{tabular}{|l|l|l|}
\hline
\hfill Variável\hfill\hspace{1mm} & \hfill Descrição${^{*}}$\hfill\hspace{1mm} & \hfill Categorização\hfill\hspace{1mm}\\
\hline
Nível de escolaridade & Número de anos de estudo (A5a, A5b, A6) & \begin{tabular}{l}Nenhum\\1 a 7 anos\\8 anos e mais\end{tabular}\\
\hline
Tem seguro/plano privado de saúde?&\hspace{-06pt}\begin{tabular}{l}Que tipo de seguro de saúde o(a) Sr.(a)\\ tem? (F1)\end{tabular} & \begin{tabular}{l}Sim\\Não\end{tabular}\\
\hline
Tem casa própria?&Esta casa é: (J2) & \begin{tabular}{l}Sim\\Não\end{tabular}\\
\hline
Uso de serviços de saúde&\hspace{-06pt}\begin{tabular}{l}Durante os últimos 12 meses, aonde o(a)\\ Sr.(a) foi quando se sentiu doente ou quando\\ precisou fazer uma consulta de saúde? (F3)\end{tabular} & \begin{tabular}{l}Usou\\Não usou\end{tabular}\\
\hline
Estado nutricional&\hspace{-06pt}\begin{tabular}{l}Com relação a seu estado nutricional o(a) \\Sr.(a) se considera bem nutrido? (C22i)\end{tabular} & \begin{tabular}{l}Bem nutrido\\Não está bem nutrido\end{tabular}\\
\hline
\end{tabular}
\scalefont{1.4184}
\end{center}
\vspace{-12pt}
Fonte: Estudo SABE.\\
\emph{${^{\textrm{*} }}$Os códigos em parênteses na descrição das variáveis se referem à identificação da variável no banco de dados do Estudo SABE.}~\cite{Gomes2011}
\end{quadro}

\chapter{Resultados e Discussões}\label{chap:RD}

\begin{table}[h]
\centering
\caption{Fatores de qualidades medidos em função do número de amostras, nos testes de reconstrução realizados.}\label{tab:qualidade}
\begin{tabular}{cc}
\toprule
Número de amostras & Fator de qualidade\\
\midrule
10 & 0.30\\
20 & 0.45\\
30 & 0.60\\
40 & 0.90\\
50 & 0.93\\
\bottomrule
\end{tabular}
\end{table}

\begin{table}[h]
\centering
\caption{Outro exemplo de tabela.}\label{tab:outroexemplo}
\begin{tabular}{ccccc}
    \toprule
    a     & b     & c     & d     & e \\
    \midrule
    10    & 20    & 30    & 40    & 50 \\
    100   & 200   & 300   & 400   & 500 \\
    \bottomrule
    \end{tabular}%
\end{table}

\chapter{Conclusão}\label{chap:Conclusao}

\renewcommand\bibname{\Large\scshape Lista de Referências}
\addcontentsline{toc}{chapter}{\bf Lista de Referências}
\bibliography{referencias}

% |--- Exemplos de Apêndices ---|----------------------{{{
% Início do Apêndice
\newcounter{apendice}
\counterwithin{figure}{apendice}
\counterwithin{table}{apendice}
\renewcommand{\theapendice}{\Alph{apendice}}
\DeclareRobustCommand{\novoapendice}[1]{%
    \refstepcounter{apendice}%
    \phantom{\theapendice}\label{#1}}

% Exemplo 1
\clearpage
\begin{flushright}
\novoapendice{apendice_exemplo}
\scalebox{1.3}{\bfseries\scshape Apêndice~\ref{apendice_exemplo}}
\addcontentsline{toc}{chapter}{Apêndice~\ref{apendice_exemplo}}
\end{flushright}

\noindent\begin{large}{\bfseries\scshape Exemplo de Apêndice}\end{large} \label{sec:apendice1}

\vspace{24pt}

Se você desejar, pode incluir ao final um ou mais apêndices, e um ou mais anexos. Caso não queira, é só remover todo o conteúdo começando na linha marcada por ``\%~Início~do~Apêndice'', até a linha anterior a ``\verb|\end{document}|''.

% Exemplo 2
\clearpage
\begin{flushright}
\novoapendice{apendice_outro_exemplo}
\scalebox{1.3}{\bfseries\scshape Apêndice~\ref{apendice_outro_exemplo}}
\addcontentsline{toc}{chapter}{Apêndice~\ref{apendice_outro_exemplo}}
\end{flushright}

\noindent\begin{large}{\bfseries\scshape Outro Exemplo de Apêndice}\end{large} \label{sec:apendice2}

\vspace{24pt}

Se você desejar, pode incluir ao final um ou mais apêndices, e um ou mais anexos. Caso não queira, é só remover todo o conteúdo começando na linha marcada por ``\%~Início~do~Apêndice'', até a linha anterior a ``\verb|\end{document}|''.
%---}}}

% |--- Exemplos de Anexos ---|----------------------{{{
% Início do anexos
\newcounter{anexo}
\counterwithin{figure}{anexo}
\counterwithin{table}{anexo}
\renewcommand{\theanexo}{\Alph{anexo}}
\DeclareRobustCommand{\novoanexo}[1]{%
    \refstepcounter{anexo}%
    \phantom{\theanexo}\label{#1}}

% Exemplo 1
\clearpage
\begin{flushright}
\novoanexo{anexo_exemplo}
\scalebox{1.3}{\bfseries\scshape Anexo~\ref{anexo_exemplo}}
\addcontentsline{toc}{chapter}{Anexo~\ref{anexo_exemplo}}
\end{flushright}

\noindent\begin{large}{\bfseries\scshape Exemplo de Anexo}\end{large} \label{sec:anexo1}

\vspace{24pt}

Se você desejar, pode incluir ao final um ou mais apêndices, e um ou mais anexos. Caso não queira, é só remover todo o conteúdo começando na linha marcada por ``\%~Início~do~Apêndice'', até a linha anterior a ``\verb|\end{document}|''.

% Exemplo 2
\clearpage
\begin{flushright}
\novoanexo{anexo_outro_exemplo}
\scalebox{1.3}{\bfseries\scshape Anexo~\ref{anexo_outro_exemplo}}
\addcontentsline{toc}{chapter}{Anexo~\ref{anexo_outro_exemplo}}
\end{flushright}

\noindent\begin{large}{\bfseries\scshape Outro Exemplo de Anexo}\end{large} \label{sec:anexo2}

\vspace{24pt}

Se você desejar, pode incluir ao final um ou mais apêndices, e um ou mais anexos. Caso não queira, é só remover todo o conteúdo começando na linha marcada por ``\%~Início~do~Apêndice'', até a linha anterior a ``\verb|\end{document}|''.
%---}}}

\end{document} 
%---}}}
