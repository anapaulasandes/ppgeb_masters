\chapter{Sincronização}

O uso de equipamentos com função exclusiva de sincronização para coletas simultâneas é comum em pesquisas ambientes academicos e clínicos.
A proposta de oferecer maior acessibilidade através da redução de custo e desenvolvimento de novas tecnologias encontra, portanto, um desafio a respeito 
de como realizar a sincronização dos dados fisiológicos sem abrir mão da praticidade e custo dos equiapementos desenvolvidos. 
Algumas propostas já foram exploradas a respeito, como o uso de piscadas e código temporal para garantir a sincronização de EEG e ET (Bækgaard et al. 2015, Notaro et al. 2018).

\section{Frequência de Coleta}
Como os sinais análogos são convertidos para sinais digitais, existe uma perda de informação por esta conversão. 
A \textbf{resolução de frequência} mede o espaço entre duas frequências. 

$$srate/N$$

Srate = sampling rate 
N = Número de amostras

\subsection{Frequência Nyquist}
É a frequência mais rápida onde o sinal pode ser medido, onde é estabelecido que a maior frequência que podemos medir é a metade 
da frequência de coleta.s


\section{Sincronização com Timecode}
Notaro et al. (2018) faz uso do código temporsal, ou \textit{timecode}, para sincronizar dados de EEG, ET e dados comportamentais 
coletados de participantes enquanto estes faziam atividades de um site de aprendizagem de linguas. O driver
do fabricante do equipamento comercial de EEG utilizado permite alteração da latência da coleta de dados, que
foi modificada do valor padrão de 16 milissegundos para 1 millisegundo, afim de aumentar a precisão do equipamento.
A informação da ocorrência de clicks no site foi retina na forma de milissegundos (HH:MM:SS:MsMsMs), e esta informação foi utilizada 
para sincronizar dados de ET, EEG e movimentação de mouse. 

\section{Sincronização com Piscadas}
Piscadas duram cerca de 200 milissegundos em média e podem indicar estados de alerta (Caffier, 2013). Piscadas também aparecem 
em dados de EEG de forma característica, podendo alcançar uma amplitude de sinal acima de 200 microvolts em eletrodos próximos a órbita ocular (Hoffmann e Falkenstein, 2008). Assim sendo, é possível realizar uma sincronização por piscadas ao se detectar 
o movimento em ambos os equiapmentos de coleta. No caso do EEG, as piscadas são comumente descartadas como artefatos indesejáveis. Já no estudo de 
Bækgaard et al. (2015), elas são a assinatura de sincronização entre os equipamentos de coleta de EEG e ET em função de sua onda característica (geralmente muitos milivolts acima do sinal do EEG), e de também 
ser detectdo através dos equipamentos de rastreamento ocular.

\subsection{Identificação no Sinal do EEG}
Para se detectar a piscada através de um sinal, é possível tentar realizar o método de \textit{Independent Component Analysis}, ou análise de 
componente independente, mas as características do sinal de piscada também permite outras abordagens, como a identificação por função de probabilidade.
Considerando o movimento de maior característica da piscada, é preferível se calcular a probabilidade do movimento de fechar os olhos 
ao movimento de abertura, em função de uma variação em tempo ser mais comumente encontrada na fase de abertura (Caffier, 2013).

\subsection{Identificação no Sinal de ET}
Como o equipamento de rastreamento procura encontrar sinais da movimentação ocular, ele também detecta a ausencia desse sinal. No estudo 
de Bækgaard et al. (2015), uma perda de até 500 milissegundos foi considerada como indicador da ocorrência de uma piscada. No equipamento de coleta de ET GP3, 
o fabricante oferece uma forma de identificar a existencia de uma piscada. Ela ocorre através da propriedade Blinking Validation Flag, ou BKID, onde qualquer 
valor diferente de 0 indica ocorrência de piscada durante o timeframe. A extração de piscada através do BKID foi utilizada no estudo de Seha et al. (2019), 
onde o blink rate foi validado e sincronizado com o vídeo do próprio equipamento (que indica quando houve piscada através da ausencia da imgem dos olhos do usuário).

\section{Correlação Cruzada}
A correlação cruzada procura calcular a similaridade entre dois sinais com a aplicação de um \textit{delay} em apenas um dos sinais.
Para correlacionar assinaturas diferentes, as probabilidades de ocorrencia de um evento (piscada) em duas séries temporais são convertidas em uma mesma frequência amostral (Bækgaard et al., 2014).
A similatidade entre sinais é medida na amplitude do sinal da correlação. A correlação cruzada é definida como:
\begin{equation}\label{eq:correlação cruzada}
    (f * g) = f(-t)*g(t), 
    \end{equation}
onde * significa convolução e f(-t) é o conjugado complexo de f(t).


\section{Códigos para Sincronização}
Alguns equipamentos podem se beneficiar da existencia de \textit{toolboxes} ou bibliotecas direcionadas à sincronização. É o caso 
dos equipamentos Tobii na solução de EEG-Eye para a linguagem MATLAB. Uma forma de se fazer sincdronização é através 
