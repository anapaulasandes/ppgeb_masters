\chapter{Sincronização}

O uso de equipamentos com função exclusiva de sincronização para coletas simultâneas é comum em pesquisas ambientes academicos e clínicos.
A proposta de oferecer maior acessibilidade através da redução de custo e desenvolvimento de novas tecnologias encontra, portanto, um desafio a respeito 
de como realizar a sincronização dos dados fisiológicos sem abrir mão da praticidade e custo dos equiapementos desenvolvidos. 
Algumas propostas já foram exploradas a respeito, como o uso de piscadas e código temporal para garantir a sincronização de EEG e ET (Bækgaard et al. 2015, Notaro et al. 2018).

\section{Sinal Elétrico}



\section{Frequência de Coleta}
Como os sinais análogos são convertidos para sinais digitais, existe uma perda de informação por esta conversão. 
A \textbf{resolução de frequência} mede o espaço entre duas frequências. A \textbf{frequência Nyquist} corresponde a maior frequência que os dados amostrados podem apresentar
sem erro e é definida como duas vezes o tamanho de banda do sinal (Leis, 2011). A formula de Nyquist é a seguinte
(desconsiderando ruído):
% John W. Leis (2011). Digital Signal Processing Using MATLAB for Students and Researchers. John Wiley & Sons. p. 82. ISBN 9781118033807.
\begin{equation}
    C(\text{bps}) = 2B * \log_2 \text{M},
\end{equation}
onde C corresponde a capacidade do sinal medida em bits por segundo (bps), B é a frequência em Hertz, e M é o 
número de níveis que o sinal pode ter. 
Isto apresenta que a frequência dos dados é proporcional a duas vezes a
 frequência de banda e logaricamente proporcional a M. 
%http://mason.gmu.edu/~rmorika2/Noise__Data_Rate_and_Frequency_Bandwidth.htm
Outra fórmula foi proposta por Shannon-Hartley, levando em consideração a proporção 
\textit{signal-to-noise} (SNR), ou relação sinal-ruído, que descreve a capacidade máxima de um sinal dado sua frequencia 
de banda. 
\begin{equation}
    C(\text{bps}) = B * \log_2 (1 + \text{SNR})
\end{equation}
A relação sinal-ruido compara a quantidade de ruído na quantidade de sinal desejado, onde um valor acima de 1 para 1
indica mais sinal que ruído. 
\begin{equation}
    \text{SNR} = \frac {P_{signal}}{P_{noise}},
\end{equation}
onde $P$ = ao poder médio. A proporção precisa ser medida dentro de uma mesma banda de frequência 
e entre mesmos pontos de um sistema. Quando o sinal é digital, o número de bits representa 
a medida e deremina o máximo do SNR. 

\section{Sincronização com Timecode}
Notaro et al. (2018) faz uso do código temporal, ou \textit{timecode}, para sincronizar dados de EEG, ET e dados comportamentais 
coletados de participantes enquanto estes faziam atividades de um site de aprendizagem de linguas. O driver
do fabricante do equipamento comercial de EEG utilizado permite alteração da latência da coleta de dados, que
foi modificada do valor padrão de 16 milissegundos para 1 millisegundo, afim de aumentar a precisão do equipamento.
A informação da ocorrência de clicks no site foi retina na forma de milissegundos (HH:MM:SS:MsMsMs), e esta informação foi utilizada 
para sincronizar dados de ET, EEG e movimentação de mouse. 

\section{Sincronização com Piscadas}
Piscadas duram cerca de 200 milissegundos em média e podem indicar estados de alerta (Caffier, 2013). Piscadas também aparecem 
em dados de EEG de forma característica, podendo alcançar uma amplitude de sinal acima de 200 microvolts em eletrodos próximos a órbita ocular (Hoffmann e Falkenstein, 2008). Assim sendo, é possível realizar uma sincronização por piscadas ao se detectar 
o movimento em ambos os equiapmentos de coleta. No caso do EEG, as piscadas são comumente descartadas como artefatos indesejáveis. Já no estudo de 
Bækgaard et al. (2015), elas são a assinatura de sincronização entre os equipamentos de coleta de EEG e ET em função de sua onda característica (geralmente muitos milivolts acima do sinal do EEG), e de também 
ser detectdo através dos equipamentos de rastreamento ocular.

O desafio da sincronização de EEG e ET se dá em função de serem séries temporais muito distintas e 
de frequências de amostra diferentes. No caso dos equipamentos comerciais de interesse desta pesquisa,
a coleta de EEG pode ser realizada em até 512 Hz, enquanto a frequencia de coleta de ET chega num máximo de 60 Hz. 
Além disso, o sinal de EEG pode apresentar mais de um canal, enquanto dados de ET podem ser representados
na forma de coordenadas. Desta forma, Bækgaard et al. (2015) propõe uma sincronização por assinaturas dentro de cada um dos tipos de dados
coletados. Piscadas ocorrem com frequencia e de forma expontanea, além de ser uma informação capturada em 
equipamentos de EEG e ET.




\subsection{Identificação no Sinal do EEG}

A piscada envolve ativação muscular, e o dipolo ocular também influencia na captura de alterações de voltagem em eletrodos próximos aos olhos
(Croft e Barry, 2000). Seu reflexo no EEG pode ser facilmente identificado pois tente a ter uma amplitude e forma de sinal característicos. 
A amplitude de uma atividade de piscada no sinal de EEG tem uma média de 200 microvolts (Hoffmann et al., 2008). Esta característica permite
que o poder elétrico somado de eletrodos de interesse possam auxiliar na determinação de uma probabilidade do evento capturado ser uma piscada. 
No estudo de Bækgaard et al. (2015), as assinaturas de piscadas foram alinhadas entre as modalidades de EEG e ET para garantir a sincronizaçã, e
o começo da atividade de piscada (com o fechamento das pálpebras) foi eleito como o ponto de referencia da assinatura. 


Para se detectar a piscada através de um sinal, é possível tentar realizar o método de \textit{Independent Component Analysis}, ou análise de 
componente independente, mas as características do sinal de piscada também permite outras abordagens, como a identificação por função de probabilidade.


\subsection{Identificação no Sinal de ET}
Como o equipamento de rastreamento procura encontrar sinais da movimentação ocular, ele também detecta a ausencia desse sinal. No estudo 
de Bækgaard et al. (2015), uma perda de até 500 milissegundos foi considerada como indicador da ocorrência de uma piscada. No equipamento de coleta de ET GP3, 
o fabricante oferece uma forma de identificar a existencia de uma piscada. Ela ocorre através da propriedade Blinking Validation Flag, ou BKID, onde qualquer 
valor diferente de 0 indica ocorrência de piscada durante o timeframe. A extração de piscada através do BKID foi utilizada no estudo de Seha et al. (2019), 
onde o blink rate foi validado e sincronizado com o vídeo do próprio equipamento (que indica quando houve piscada através da ausencia da imgem dos olhos do usuário).



\section{Métodos para Sincronização}
Existem diferentes métodos para calculo de sincronização entre séries temporais. 
A forma mais simples é por \textbf{correlação de Pearson}, que mede como dois sinais mudam ao longo do tempo em valores que vão de -1 a 1;
com -1 indicando uma correlação perfeita e negativa, 1 uma correlação perfeita e 0, sem correlação. 
É importante notar que anomalias vão impactar significativamente a correlação, e que os dados assumem que a variância é homogênea. 

Outro método de sincronização entre séries temporais é o \textit{\textbf{lag cross-correlation}}, onde é possível identificar
qual sinal vem primeiro. A correlação é calculada através da mudança gradual de um vetor de série temporal e subsequente cálculo 
de correlação. A correlação cruzada procura calcular a similaridade entre dois sinais com a aplicação de um \textit{delay} em apenas um dos sinais.

Um exemplo de seu uso é no trabalho de Com dois sinais diferentes em EEG e ET, Bækgaard et al. (2015) opta por correlacionar as funções de probabilidade do evento observado em 
ambos os equipamentos, ser uma piscada. 
Para correlacionar assinaturas diferentes, as probabilidades de ocorrencia de um evento (piscada) em duas séries temporais são convertidas em uma mesma frequência amostral (Bækgaard et al., 2014).
A similatidade entre sinais é medida na amplitude do sinal da correlação. A correlação cruzada é definida como:
\begin{equation}\label{eq:correlação cruzada}
    (f * g) = f(-t)*g(t), 
    \end{equation}
onde * significa convolução e f(-t) é o conjugado complexo de f(t).






\subsection{Códigos para Sincronização}
Alguns equipamentos podem se beneficiar da existencia de \textit{toolboxes} ou bibliotecas direcionadas à sincronização. É o caso 
dos equipamentos Tobii na solução de EEG-Eye para a linguagem MATLAB. Uma forma de se fazer sincdronização é através 

\section{Correlação EEG e ET}

 Shared triggers
Common trigger pulses ("triggers") are sent frequently from the 
stimulation computer to both ET computer and EEG recording computer. 
This is achieved via a Y-shaped cable that is attached to the parallel 
port of the stimulation computer and splits up the pulse so it is looped through 
to EEG and ET. We recommend to send triggers with a sufficient duration 
(e.g. at least 5 ms at 500 Hz sampling rate) to avoid the loss of some of the triggers.
 The advantage of this method is that the same physical signal is used for 
 synchronization (although this does not guarantee that the trigger is
  inserted into the ET and EEG data streams without delays). The disadvantage
   is the need for an extra cable.

Messages+triggers
Messages are short text strings that can be inserted into the eye tracking data.
 While triggers are still sent to the EEG, messages are used as the corresponding 
 events for the ET. Here, the ET computer is given a command to insert an ASCII text 
 message (containing a keyword and the value of the corresponding EEG trigger) 
 into the eye tracking data. In the stimulation software, the commands to send a trigger
  (to the EEG) and a message (to the ET) are given in immediate succession. 

  Analogue output
  A copy of the eye track is fed directly into the EEG. A digital-to-analogue 
  converter card in the ET outputs (some of) the data as an analogue signal.
   With SMI, this signal can be fed directly into the EEG headbox. 
   This requires a custom cable and resistors to scale the output voltage 
   of the D/A converter to the EEG amplifier's recording range. While this
    method affords easy synchronisation, there are disadvantages: First, 
    voltages need to be rescaled to pixels for analysis. Second, the ET 
    signal may exceed the amplifier’s recording range and electrical 
    interference with the EEG is possible. 
    Third, additional information from the ET 
    (messages, eye movements detected online) is 
    not available. Fourth, quality of the ET signal 
    suffers considerably from the D/A and subsequent 
    A/D conversion. Finally, fewer channels remain to record
     the EEG (recording binocular gaze position and pupil diameter occupies six channels).

     Basics: Synchronization signals

     Send triggers and/or messages to align the recordings
The toolbox requires that there are at least two shared events present in the ET and EEG: One near the beginning and one near the end of the recording. These events will be called start-event and end-event in the following. Eye tracking data in between the start-event and end-event will be linearly interpolated to match the sampling frequency of the EEG. We recommend to use a unique event value (e.g. "100") to mark the start-event and another unique event-value (e.g., "200") for the end-event. The remaining shared events (triggers or messages) sent during the experiment (between start-event and end-event) are used to evaluate the quality of synchronization. Synchronization is possible even if some intermediate events were lost during transmission.

Original sampling rates of EEG and ET do not need to be the same. The ET will be resampled to the sampling frequency of the EEG. For example, if the EEG was sampled at 500 Hz, and eye movements were recorded at 1000 Hz, the toolbox will downsample the eye track to 500 Hz. Since ET data outside of the synchronization range (before start-event, after end-event) cannot be interpolated, it is replaced by zeros. Please note that the EEG recording should not be paused during the experiment. [Clarification, April 2013: It is not a problem to pause the eye tracker recording, e.g. for recalibrations, because the time stamp assigned to each ET sample continues to increase even during the pause. However, the toolbox currently cannot recognize and handle pauses in the EEG recording. Therefore, the EEG recording should be continuous and must not be paused.]

Note:The current Beta version of the toolbox does not yet implement low-pass filtering of the eye track to prevent aliasing in case that the eye track is downsampled to a much lower EEG sampling rate. We plan to add this in the future.

If synchronization method 2 (messages plus trigger) is used, synchronization messages sent to the eye tracker need to have a specified format. This format consists of an arbitrary user-defined keyword (e.g., "MYKEYWORD") followed by an integer value ("MYKEYWORD 100"). The integer value needs to be the same as that of the corresponding trigger pulse sent to the EEG (usually an 8-bit number between 1 and 255). An example is given in the code below. The EYE-EEG parser (Step 2: Preprocess eye track and store as MATLAB) will recognize messages with the keyword and treat them as synchronization events. A keyword-synchronization messages should be sent together with every trigger sent to the EEG, so intermediate events in-between start-event and end-event can be used to assses synchronization quality. Additional messages (that do not contain the keyword) may be sent to code other aspects of the experimental design. They are ignored by the toolbox.

The following code is an example for an experimental runtime file containing the necessary synchronization signals. The example is for the software Presentation™, but similar commands exist in other software (e.g., Psychtoolbox, EPrime™):++


