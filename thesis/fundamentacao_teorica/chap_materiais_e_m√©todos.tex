\chapter{Materiais e Métodos}

\section{Proposta de Ferramenta}

Dois equipamentos serão utilizados para construção da ferramenta de coleta sincronizada: Mindwave Mobile II da Neurosky Inc. e o
GP3 da Gazepoint Inc., para a coleta de EEG e ET, respectivamente. Um código será construído de forma a permitir a coleta 
síncrona do movimento ocular e das ondas cerebrais pelos dois equipamentos comerciais. 
No presente capítulo as ferramentas serão apresentadas no que diz respeito às suas principais características, 
e especificidades do código de coleta serão apresentadas.

\section{Mindwave Mobile II}
O equipamento (representado na figura 6.1) possui um eletrodo de coleta e um eletrodo de referência que ficam posicionados 
acima da sobrancelha esquerda e na orelha esquerda do participante, respectivamente. 
A posição do eletrodo de coleta em relação ao sistema de referência de posição de eletrodos (10-20), é o FP1, 
correspondendo a região Frontopolar 1. A coleta de dados do aparelho se dá por conexão via bluetooth e 
funciona em computadores Mac, Windows ou celulares Androids ou iOS, disponíveis em um raio de 10 metros. 
Ele coleta ondas cerebrais variando entre 3 e 100Hz, com uma frequência de 512Hz (NeuroSky Inc., 2015). 
O aparelho automaticamente distingue os dados coletados em ondas alfa, beta, gama, teta e delta; 
além de coletar informações subjetivas no formato de medidas de atenção e meditação, 
por meio de um algoritmo de reforço de aprendizado não disponibilizado ao publico. 
Também mede a ativação muscular próxima ao eletrodo para estimar a qualidade do sinal. 
O MindWave Mobile filtra interferência elétrica e converte o sinal detectado pelo eletrodo em sinal digital. 
O chip que faz o filtro e conversão se chama ThinkGear, e permite a filtragem de ruído para interferência ativação muscular (EMG) e 50/60Hz de corrente alternada. 

\section{Gazepoint GP3}

O GP3 é um equipamento comercial de coleta do movimento dos olhos, 
fabricado pela Gazepoint Inc (representado na figura 6.2). Possui software próprio para análise dos dados, 
além de ser possível realizar coleta de dados com linguagens de programação open-source. O GP3 funciona emitindo uma luz infravermelha (IR) 
diretamente nos olhos do participante e captando a reflexão da luz para localizar o ponto focal ao longo do tempo. 
Permite coletar a direção do olhar, número de fixações, tempo até a primeira fixação, taxa de piscadas,
 duração de piscadas, diâmetro da pupila, tempo de duração do olhar em um determinado ponto focal, 
 objetos observados em uma imagem, entre outros (Gazepoint Inc.).


\subsection{Especificações GP3}

O Gazepoint GP3 estabelece sua conexão com o computador através de dois cabos USB - um cabo de energia e outro para dados.
Seu posicionamento ideal é logo abaixo do monitorde estímulo. Para um melhor posicionamento, o fabricante 
sugere uma distância ideal de 65 cm dos olhos do participante até o equipamento. O GP3 possui as seguintes características:

\begin{itemize}
    \item Acurácia de 0.5-1 grau de ângulo visual
    \item 60 Hz de frequencia de atualização
    \item calibração de 5 e 9 pontos
    \item API
    \item Captura movimento de 25cm horizontais e 11cm verticais
    \item 15 cm de limite de profundidade de movimento
\end{itemize}

Para poder realizar a coleta dos dados, é necessário manter o Gazepoint Control (API do desenvolvedor) ligado. 

\subsection{Calibração GP3}
Uma calibração é realizada pela própria API do equipamento, afim de estabelecer qual o apontamento ocular do participante. 
A calibração pode ser feita em 5 pontos ou 9 pontos no monitor de exibição de estímulo. Os pontos na tela são apresentados
em sequencia e o participante deve acompanha-los com o olhar até a finalização da calibração. 

Após a calibração ser concluída, a API calcula o erro do sistema em relação ao olhar para o olho esquerdo (em verde) 
e direito (vermelho). 

\subsection{Dados Capturados pelo GP3}

\textbf{Fixação} É um agrupamento de pontos focais do olhar que duram entre 20-300 ms (Brand, 2020).

\textbf{Gaze Point} Gaze point é o ponto focal do usuário em um dado momento. No equipamento GP3 é gravado um ponto focal a cada aproximadamente 17 milisegundos.
O ponto de gaze é gravado em relação as coordenadas x e y, que servem para identificar a posição do olhar na tela de experimento. 

\textbf{Sacada} A sacada compreende a um movimento rápido dos olhos após a fixação.

\subsection{Avaliação da Qualidade dos Dados}
\textbf{Acurácia}
\textbf{Precisão}



\subsection{}