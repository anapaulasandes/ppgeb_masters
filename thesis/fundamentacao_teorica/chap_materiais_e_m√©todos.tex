\chapter{Materiais e Métodos}

\section{Proposta de Ferramenta}

Dois equipamentos serão utilizados para construção da ferramenta de coleta sincronizada: Mindwave Mobile II da Neurosky Inc. e o
GP3 da Gazepoint Inc., para a coleta de EEG e ET, respectivamente. Um código será construído de forma a permitir a coleta 
síncrona do movimento ocular e das ondas cerebrais pelos dois equipamentos comerciais. 
No presente capítulo as ferramentas serão apresentadas no que diz respeito às suas principais características, 
e especificidades do código de coleta serão apresentadas.

\section{Mindwave Mobile II}
O equipamento (representado na figura 6.1) possui um eletrodo de coleta e um eletrodo de referência que ficam posicionados 
acima da sobrancelha esquerda e na orelha esquerda do participante, respectivamente. 
A posição do eletrodo de coleta em relação ao sistema de referência de posição de eletrodos (10-20), é o FP1, 
correspondendo a região Frontopolar 1. A coleta de dados do aparelho se dá por conexão via bluetooth e 
funciona em computadores Mac, Windows ou celulares Androids ou iOS, disponíveis em um raio de 10 metros. 
Ele coleta ondas cerebrais variando entre 3 e 100Hz, com uma frequência de 512Hz (NeuroSky Inc., 2015). 
O aparelho automaticamente distingue os dados coletados em ondas alfa, beta, gama, teta e delta; 
além de coletar informações subjetivas no formato de medidas de atenção e meditação, 
por meio de um algoritmo de reforço de aprendizado não disponibilizado ao publico. 
Também mede a ativação muscular próxima ao eletrodo para estimar a qualidade do sinal. 
O MindWave Mobile filtra interferência elétrica e converte o sinal detectado pelo eletrodo em sinal digital. 
O chip que faz o filtro e conversão se chama ThinkGear, e permite a filtragem de ruído para interferência ativação muscular (EMG) e 50/60Hz de corrente alternada. 

\section{Gazepoint GP3}

O GP3 é um equipamento comercial de coleta do movimento dos olhos, 
fabricado pela Gazepoint Inc (representado na figura 6.2). Possui software próprio para análise dos dados, 
além de ser possível realizar coleta de dados com linguagens de programação open-source. O GP3 funciona emitindo uma luz infravermelha (IR) 
diretamente nos olhos do participante e captando a reflexão da luz para localizar o ponto focal ao longo do tempo. 
Permite coletar a direção do olhar, número de fixações, tempo até a primeira fixação, taxa de piscadas,
 duração de piscadas, diâmetro da pupila, tempo de duração do olhar em um determinado ponto focal, 
 objetos observados em uma imagem, entre outros (Gazepoint Inc.).


\subsection{Especificações GP3}

O Gazepoint GP3 estabelece sua conexão com o computador através de dois cabos USB - um cabo de energia e outro para dados.
Seu posicionamento ideal é logo abaixo do monitorde estímulo. Para um melhor posicionamento, o fabricante 
sugere uma distância ideal de 65 cm dos olhos do participante até o equipamento. O GP3 possui as seguintes características, conforme
especificado pelo fabricante:

\begin{itemize}
    \item Acurácia de 0.5-1 grau de ângulo visual
    \item 60 Hz de frequencia de atualização
    \item calibração de 5 e 9 pontos
    \item API
    \item Captura movimento de 25cm horizontais e 11cm verticais
    \item 15 cm de limite de profundidade de movimento
\end{itemize}

Para poder realizar a coleta dos dados, é necessário manter o Gazepoint Control (API do desenvolvedor) ligado. 

\subsection{Calibração GP3}
Uma calibração é realizada pela própria API do equipamento, afim de estabelecer qual o apontamento ocular do participante. 
A calibração pode ser feita em 5 pontos ou 9 pontos no monitor de exibição de estímulo. Os pontos na tela são apresentados
em sequencia e o participante deve acompanha-los com o olhar até a finalização da calibração. 

Após a calibração ser concluída, a API calcula o erro do sistema em relação ao olhar para o olho esquerdo (em verde) 
e direito (vermelho). 

\subsection{Dados Capturados pelo GP3}

\textbf{Fixação:} É um agrupamento de pontos focais do olhar que duram entre 20-300 ms (Brand, 2020).

\textbf{Gaze Point:} Gaze point é o ponto focal do usuário em um dado momento. No equipamento GP3 é gravado um ponto focal a cada aproximadamente 17 milisegundos.
O ponto de gaze é gravado em relação as coordenadas x e y, que servem para identificar a posição do olhar na tela de experimento. 

%Anjith George and Aurobinda Routray, “A score level fusion method for eye movement biometrics,” Pattern Recognition Letters, vol. 82, pp. 207–215, 2016
\textbf{Sacada:} Compreende a um movimento rápido dos olhos após a fixação, e pode ser medida através de pixels por segundo.
O valor limite entre sacada e fixação é a velocidade de 1.8 pixels por segundo (George e Routray, 2016), onde acima
é uma sacada e abaixo, uma fixação.



\subsection{Avaliação da Qualidade dos Dados}

\textbf{Acurácia} é a capacidade de medir a localização do olhar do usuário na tela. 

\textbf{Precisão} é a habilidade de produzir movimentos consistentes. 

\section{Montagem}

6.5 MONTAGEM 
Para o uso do GP3, o software próprio do fabricante será utilizado, o Gazepoint Control. Para o uso do brainwave mobile 2, um aplicativo Windows será utilizado para atestar a conexão e apresentar a porta de saída da conexão bluetooth (porta COM), o NeuroExperiementer. Para a coleta simultânea, duas linguagens de programação serão utilizadas: MATLAB e Python, com o uso do Anaconda para gerenciamento de pacotes de interesse, no caso do desenvolvimento em Python. 
 A montagem do setup será feita com base nas coletas realizadas para a construção dos datasets EEGEyeNet e ZuCO, com o participante sentado frente a tela de apresentação de estímulos, e com os olhos posicionados a uma distância ideal do aparelho de. ET. O Gazepoint ficará distante do participante até o software próprio acusar distância ideal (sinalizado por um sinal verde no topo do software de regulação do equipamento). Para teste de coleta serão utilizados dois monitores: um para calibração e apresentação de imagens, e outro para o desenvolvimento de código e testagem (figura 6.3). Após verificar que ambos os equipamentos estão conectados ao computador, o ambiente de desenvolvimento será ativado por terminal e a variável com o número de porta COM correspondente a atual conexão via bluetooth do Mindwave será alterado no respectivo código de teste. O código será então executado.
 
Figura 6.3 Montagem de Coleta. Fonte: Autoria Própria.





Tabela 6.1 Caracteristicas Montior utilizado na Coleta de Eye Tracking 
Monitor	Video Interno conectado a Intel® HD Graphics 630
Resolução da área de trabalho	1920  x 1080
Resolução do sinal ativo	1920  x 1080
Taxa de atualização	60 Hz
Intensidade de bits	6 bits
Formato de cor 	RGB
Espaço de cores	Alcance dinâmico padrão

6.6 CRIAÇÃO E SINCRONIZAÇÃO DO DATASET
O equipamento GP3 possui uma Application Programming Interface (API) própria, o Open Gaze API (Gazepoint ®). A API é uma alternativa de se controlar o equipamento sem precisar do software pago tambem feito pela Gazepoint ®. A API utiliza de um Transmission Control Protocol – Internet Protocol (TCP/IP) socket, que permite a comunicação entre a aplicação e o servidor (fonte dos dados de ET). O IP determina o endereço para o qual os dados serão enviados, e o TCP utiliza a arquitetura de rede para realizar o transporte. O formato de dado utilizado para a API é o Extensible Markup Language (XML), e pode ser implementado em outras linguagens, como Python. As portas utilizadas para a comunicação de forma padrão são: localhost (127.0.0.1) e port 4242. 

Estas portas e a linguagem XML serão utilizadas para possibilitar o uso da linguagem Python e MATLAB para estabelecer a conexão e iniciar as principais funções da ferramenta GP3. No API o cliente tem duas tags de comunicação: GET e SET. Ao utilizar o SET o cliente pode alterar o valor de alguma variável. O comando GET não tem a possibilidade de alterção de valores. sO servidor pode enviar dados para o cliente com diferentes tags: ACK, NACK, CAL e REC. As duas primeiras são geradas em resposta aos comandos de GET e SET (ACK – Sucesso e NACK –Falha). Os CAL são gerados com bases nas calibrações e REC serve para os dados gravados. Estas regras de escrita são utilizadas pelo codigo em Python para estabelecer o controle do GP3, permitindo que o código calibre o equipamento e estabeleça definições de variáveis.

Para o funcionamento adequado do Mindwave, é necessário instalar uma tecnologia chamada ThinkGear, que permite a troca de informações entre o equipamento e os softwares compativeis e processa o sinal detectado pelo eletrodo. O ThinkGear também é responsável pelo cálculo dos chamados eSense Meteres, correspondendo aos dados de Atenção e Meditação (algoritmo não apresentado), em uma escalda de 0 a 100. Sobre variações individuais, a Mindwave ressalta que: “algumas partes do algoritmo do eSense estao aprendendo dinamicamente e aplicando algoritmos de aprendizado de adaptaçao lenta para ajustar a flutuações naturais de cada individuo (...) permitindo que o sensor opere em condições pessoais e ambientais diversas, enquanto ainda oferece uma boa acuracia e confiabilidade.”

Para a operação do Mindwave, também é necessário informar a frequência elétrica do país (no Brasil, 60Hz, mesmo que Estados Unidos e Canadá), o que o fabricante indica ser realizado pelo software Mindwave Mobile Tutorial. Esta frequencia é definida com a inteção de ser removida dos dados coletados pelo Mindwave Mobile II, na forma de preprocessamento para remoção de artefatos (Mindwave®, 2018). 

Após essas configurações iniciais o código será iniciado. A construção de código seguirá dois propósitos de testes de sincronização: por timecode e por coleta controlada por código. Uma representação da conexão do código com o equipamento é apresentada na figura 6.4, e um exemplo de construção de dataset pelo método citado acima é apresentado na figura 6.5. 



\subsection{}