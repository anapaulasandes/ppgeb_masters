\chapter{Acesso a Datasets Fisiológicos}

Apesar da existência de fontes de datasets fisiológicos, 
o número de datasets disponíveis ainda é restrito. Sobre a qualidade dos datasets, 
Mendoza et al. (2021) observou que nenhum dos nove datasets disponíveis publicamente para 
treinamento de algoritmos de classificação de emoção analisados possuía todos os critérios de 
referência levantados por estudos anteriores. Apesar da ausência dessas referencias, os datasets 
apresentaram uma base para desenvolvimentos futuros. Sobre a disponibilidade de datasets fisiológicos,
Rim et al. (2020) faz uma análise de datasets públicos e privados.
 No exemplo apresentado na figura 3.1, 
 é possível observar que datasets públicos combinando sinais são minoria nas diferentes fontes de dados analisadas. 

 O campo de neurociência computacional é um dos diversos campos beneficiados 
 com o desenvolvimento da tecnologia, que constantemente melhora no sentido 
 de propor novas ferramentas de captura de sinais fisiológicos e novas formas de
  processá-los. Diferentes equipamentos de EEG e ET implicam em uma diferente forma
   de se montar a coleta e processar os dados coletados.
  Embora surjam novos métodos, importantes considerações devem ser feitas a 
  respeito da resolução de captura, de forma a não se deixar perder informação desejada.   

\section{Aquisição de Dados Fisiológicos}

Um exemplo de como se realizar a montagem para coleta de EEG e ET é demonstrado na figura 3.1, utilizado na montagem do dataset EEGEyeNet (Kastrati et al., 2021),
onde o participante é colocado de frente para o monitor para apresentação de estímulos com o equi
pamento de coleta de EEG sobre a cabeça e o aparelho de ET direcionado aos olhos do participante. A piscada é comumente
removida como artefato indesejável nos dados de EEG, fazendo parte de muitos pré-processamentos de estudos com EEG e ET (Hosseini, 2020).
Entretanto, ET e EEG podem ser sincronizados a partir da assinatura da piscada, permitindo uma correção contínua dos dados (Bækgaard e Larsen, 2014). 
Outras formas de sincronização de EEG e ET também foram propostas, como por código temporal ou com auxílio de equipamentos externos – exploradas adiante. 
 
Figura 3.2. Setup de coleta de EEG e ET. Fonte: Kastrati et al. (2021)
A fusão de dados pode ser realizada no nível de característica ou feature,
assim como a nível de decisão do algoritmo classificatório (Klein, 2014; Mendes et al., 2016).  explicadas adiante.
      Na figura 3.2. é possível observar um fluxo de processamento dos sinais capturados por diferentes sensores.  Os diferentes 
      sensores representados no estudo de Mendes et al. (2016) são aqui representados pela coleta de EEG e ET.
 

Figura 3.2. Exemplo de Fluxo para Fusão de Dados de Sensores. Fonte:  Mendes et al. (2016).
Na Feature Level Fusion (FLF), os dados de diversas fontes são extraídos dos sensores e unidos de forma a 
gerar um vetor único com informações multimodais; no Decision Fusion (DL) a classificação ocorre para cada categoria 
de fonte de dado (exemplo: uma classificação para EEG e outra para ET) e estas classificações são combinadas em um esquema de voto (exemplo: a classificação mais comum) para se chegar em uma categoria final (Bota et al., 2020). A respeito de qual formato seria melhor, Bota et al. (2020) explorou o assunto para cinco bases de dados fisiológicos classificados de acordo com o estímulo emocional apresentado ao participante e observou que o melhor método de fusão é altamente correlacionado à base dados, embora o FLF tenha sido escolhido como o melhor em função de sua baixa complexidade em relação ao DF. 
No estudo de Kastrati et al. (2021), dados de EEG e ET foram coletados por equipamentos com 500Hz de resolução e si
ncronizados por código, com auxílio do Eye EEG Toolbox para MATLAB. A sincronização foi confirmada pelo início de s
inais registrados em ambos os equipamentos, apresentando erros menores que dois milissegundos. 
